% Customizable fields and text areas start with % >> below.
% Lines starting with the comment character (%) are normally removed before release outside the collaboration, but not those comments ending lines

% svn info. These are modified by svn at checkout time.
% The last version of these macros found before the maketitle will be the one on the front page,
% so only the main file is tracked.
% Do not edit by hand!
\RCS$Revision: 279828 $
\RCS$HeadURL: svn+ssh://nfilipov@svn.cern.ch/reps/tdr2/notes/HIN-15-001/trunk/HIN-15-001.tex $
\RCS$Id: HIN-15-001.tex 279828 2015-03-07 11:09:52Z raphael $
%%%%%%%%%%%%% local definitions %%%%%%%%%%%%%%%%%%%%%
% This allows for switching between one column and two column (cms@external) layouts
% The widths should  be modified for your particular figures. You'll need additional copies if you have more than one standard figure size.
\newlength\cmsFigWidth
\ifthenelse{\boolean{cms@external}}{\setlength\cmsFigWidth{0.85\columnwidth}}{\setlength\cmsFigWidth{0.4\textwidth}}
\ifthenelse{\boolean{cms@external}}{\providecommand{\cmsLeft}{top\xspace}}{\providecommand{\cmsLeft}{left\xspace}}
\ifthenelse{\boolean{cms@external}}{\providecommand{\cmsRight}{bottom\xspace}}{\providecommand{\cmsRight}{right\xspace}}

% Raph : private new commands
\newcommand {\npart}  {\ensuremath{N_{\text{part}}}\xspace}
\newcommand {\ncoll}  {\ensuremath{N_{\text{coll}}}\xspace}
\newcommand{\raa}{\ensuremath{R_{AA}}\xspace}
\newcommand{\taa}{\ensuremath{T_{AA}}\xspace}
\newcommand{\sqrts}{\ensuremath{\sqrt{s}}\xspace}
\newcommand{\sqrtsnn}{\ensuremath{\sqrt{s_{_{\text{NN}}}}}\xspace}
\newcommand{\QQbar}{Q\ensuremath{\overline{\textrm{Q}}}}

%%%%%%%%%%%%%%%  Title page %%%%%%%%%%%%%%%%%%%%%%%%
\cmsNoteHeader{HIN-15-001} % This is over-written in the CMS environment: useful as preprint no. for export versions
% >> Title: please make sure that the non-TeX equivalent is in PDFTitle below
\title{Suppression of $\Upsilon$(1S), $\Upsilon$(2S) and $\Upsilon$(3S) in PbPb collisions at $\sqrt{s_{\rm NN}} = 2.76$~TeV}

% >> Authors
%Author is always "The CMS Collaboration" for PAS and papers, so author, etc, below will be ignored in those cases
%For multiple affiliations, create an address entry for the combination
%To mark authors as primary, use the \author* form
%\address[neu]{Northeastern University}
%\address[fnal]{Fermilab}
%\address[cern]{CERN}
\author[cern]{The CMS Collaboration}

% >> Date
% The date is in yyyy/mm/dd format. Today has been
% redefined to match, but if the date needs to be fixed, please write it in this fashion.
% For papers and PAS, \today is taken as the date the head file (this one) was last modified according to svn: see the RCS Id string above.
% For the final version it is best to "touch" the head file to make sure it has the latest date.
\date{\today}

% >> Abstract
% Abstract processing:
% 1. **DO NOT use \include or \input** to include the abstract: our abstract extractor will not search through other files than this one.
% 2. **DO NOT use %**                  to comment out sections of the abstract: the extractor will still grab those lines (and they won't be comments any longer!).
% 3. For PASs: **DO NOT use tex macros**         in the abstract: CDS MathJax processor used on the abstract doesn't understand them _and_ will only look within $$. The abstracts for papers are hand formatted so macros are okay.
\abstract{
Yields of $\Upsilon$(1S), $\Upsilon$(2S) and $\Upsilon$(3S)
mesons, are measured in the CMS experiment via their $\mu^+ \mu^-$
decays for bottomonium rapidity $|y|<2.4$, and in PbPb and pp collisions 
of similar nucleon-nucleon collision luminosity and same energy,  
$\sqrt{s_{\rm NN}} = 2.76$~\TeV . Differential cross sections and
nuclear modification factors are reported as functions of $y$ and
transverse momentum \pt, as well as collision centrality. 
A strong, centrality-dependent suppression is observed in PbPb
collisions, compared to the yield in pp collisions scaled by the
number of inelastic nucleon-nucleon collisions. 
}

% >> PDF Metadata
% Do not comment out the following hypersetup lines (metadata). They will disappear in NODRAFT mode and are needed by CDS.
% Also: make sure that the values of the metadata items are sensible and are in plain text:
% (1) no TeX! -- for \sqrt{s} use sqrt(s) -- this will show with extra quote marks in the draft version but is okay).
% (2) no %.
% (3) No curly braces {}.
\hypersetup{%
pdfauthor={Nicolas Filipovic, Chad Flores, Francois Arleo, Manuel Calderon de la Barca Sanchez, Emilien Chapon, Raphael Granier de Cassagnac, Prashant Shukla},%
pdftitle={Nuclear modification of Y states in PbPb},%
pdfsubject={CMS},%
pdfkeywords={CMS, physics, heavy ions, dimuons, bottomonia}}

\maketitle %maketitle comes after all the front information has been supplied
% >> Text
%%%%%%%%%%%%%%%%%%%%%%%%%%%%%%%%  Begin text %%%%%%%%%%%%%%%%%%%%%%%%%%%%%
%% **DO NOT REMOVE THE BIBLIOGRAPHY** which is located before the appendix.
%% You can take the text between here and the bibiliography as an example which you should replace with the actual text of your document.
%% If you include other TeX files, be sure to use "\input{filename}" rather than "\input filename".
%% The latter works for you, but our parser looks for the braces and will break when uploading the document.
%%%%%%%%%%%%%%%

\section{Introduction} \label{sec:intro}

At large energy densities and high temperatures, strongly interacting
matter consists of a deconfined and chirally-symmetric system of
quarks and gluons~\cite{Karsch:2003jg}. This state, often referred to
as ``quark-gluon plasma'' (QGP)~\cite{Shuryak:1977ut}, constitutes the
main object of the studies performed with relativistic heavy-ion
collisions.

The formation of a QGP in high-energy nuclear collisions can be
evidenced in a variety of ways. One of its most striking expected
signatures is the suppression of quarkonium
states~\cite{Matsui:1986dk}, both of the charmonium (\JPsi, $\psi'$,
$\chi_{c}$, etc.) and the bottomonium ($\PgU\text{(1S,\,2S,\,3S)}$,
$\chi_{b}$, etc.) families.  This is thought to be a direct effect of
deconfinement, when the binding potential between the constituents of
a quarkonium state, a heavy quark and its antiquark, is screened by
the colour charges of the surrounding light quarks and gluons. The
suppression is predicted to occur above the critical temperature of
the medium ($T_{c}$) and depends on the \QQbar~binding energy. Since the
\PgUa\ is the most tightly bound state among all quarkonia, it is
expected to be the one with the highest dissociation
temperature. Examples of dissociation temperatures are given in
Ref.~\cite{Mocsy:2007jz}: $T_{\text{dissoc}}\sim\!1\,T_{c}$, $1.2\,T_{c},$
and $2\,T_{c}$ for the \PgUc, \PgUb, and \PgUa, respectively.
However, there are further possible changes to the
quarkonium production in heavy-ion collisions, such as
modifications to the parton distribution functions inside the nucleus
(shadowing) and other cold-nuclear-matter effects that are expected to 
reduce the production of quarkonia without the presence of a
QGP~\cite{Vogt:2010aa}.

The suppression of the \PgUa in heavy-ion collisions was first measured by the CMS 
collaboration~\cite{Chatrchyan:2012np}. 
The suppression of the \PgUb\ and \PgUc\ was also first hinted~\cite{Chatrchyan:2011pe} then observed~\cite{Chatrchyan:2012lxa} by CMS. 
In pPb collisions, the ALICE~\cite{Abelev:2014oea} and LHCb~\cite{Aaij:2014mza} collaboration reported yields that are compatible, or slightly lower, than the binary-scales yields extrapolated from pp collisions at different energies. The \PgUb\ and \PgUc\ were reported to be slightly more suppressed than the \PgUa~\cite{Chatrchyan:2013nza}.

In this document, the yields of \PgUa, \PgUb\ and \PgUc\ mesons are reported, from a PbPb and a pp data sets corresponding to 150~\mubinv and 5.4~\pbinv, respectively. With respect to~\cite{Chatrchyan:2012lxa}, the PbPb data reconstruction was improved yielding to a raise of $\approx 30$\% in the \PgUa\ yields. The pp data set, recorded in 2013, is new and permits  further differential studies as functions of the \PgU\ $y$ and \pt. From the PbPb and pp yields, nuclear modification factors, $R_{\rm AA}$, are derived. 

% Raph : 1784 Y(1S) / 1380 in AN2011_455, tight cuts

% Raph : pas de plan...

\section{The CMS detector} \label{sec:detector}

A detailed description of the CMS experiment can be found in
Ref.~\cite{Adolphi:2008zzk}. The central feature of the CMS apparatus
is a superconducting solenoid of 6\,m internal diameter. Within the
field volume are the silicon tracker, the crystal electromagnetic
calorimeter, and the brass/scintillator hadron calorimeter.

Muons are detected in the interval $|\eta|< 2.4$ by gaseous detectors
made of three technologies: drift tubes, cathode strip chambers, and
resistive plate chambers, embedded in the steel return yoke.  The
silicon tracker is composed of pixel detectors (three barrel layers
and two forward disks on either side of the detector, made of
66~million $100\times150\mum^2$ pixels) followed by microstrip
detectors (ten barrel layers plus three inner disks and nine forward
disks on either side of the detector, with strips of pitch between 80
and 180\mum). The transverse momentum of muons matched to
reconstructed tracks is measured with a resolution better than
${\sim}1.5$\% for \pt smaller than 100\GeVc~\cite{TRK-10-004}.  The
good resolution is the result of the 3.8\,T magnetic field and the
high granularity of the silicon tracker.

In addition, CMS has extensive forward calorimetry, including two
steel/quartz-fibre Che\-ren\-kov forward hadron (HF) calorimeters,
which cover the pseudorapidity range $2.9 < |\eta|< 5.2$. These
detectors are used in the present analysis for the event selection and
PbPb collision centrality determination, as described in the next
section. 

\section{Selections}

\subsection{Event selection and centrality}

In order to select a sample of purely inelastic hadronic PbPb collisions, the contamination from ultraperipheral collisions and non-collision beam background is removed, as described in Ref.~\cite{Chatrchyan:2011sx}. Events are preselected if they contain a reconstructed primary vertex containing at least two tracks and at least three HF towers on each side of the interaction point with an energy of at least 3\GeV deposited in each tower. To further suppress the beam-gas events, the distribution of hits in the pixel detector along the beam direction is required to be compatible with particles originating from the event vertex. These criteria select $(97\pm3)\%$ of hadronic PbPb collisions~\cite{Chatrchyan:2011sx}, corresponding to a number of efficiency-corrected minimum bias (MB) events $N_\mathrm{MB}=(1.16\pm0.04)\times10^9$ for the sample analyzed. 
The pp data set corresponds to an integrated luminosity of 5.4\pbinv known to an accuracy of 3.7\% from the uncertainty in the calibration based on a van der Meer scan~\cite{CMS-PAS-LUM-13-002}.

The measurements reported here are based on dimuon events triggered by
the \Lone (L1) trigger, a hardware-based trigger that uses information
from the muon detectors. No selection is applied on the muon momentum. 
The CMS detector is also equipped with a software-based high-level trigger (HLT). 
It is there required that the L1 muons are of high quality. The pp and PbPb data
follow the same trigger logic. 

The centrality variable is defined as the
fraction of the total cross section, starting at 0\% for the most
central collisions. This fraction is determined from the distribution
of total energy measured in both HF
calorimeters. Using a Glauber-model
calculation as described in Ref.~\cite{Chatrchyan:2011sx}, one can
estimate variables related to the centrality, such as the number of
nucleons participating in the collisions (\npart) and the nuclear
overlap function (\taa), which is equal to the number of elementary
nucleon-nucleon (NN) binary collisions divided by the elementary NN
cross section and can be interpreted as the NN equivalent integrated
luminosity per heavy ion collision, at a given
centrality~\cite{Miller:2007ri}. The values of these variables are
presented in \ref{tab:glauber} for the centrality bins used in this
analysis. In the following, \npart will be the variable used to show
the centrality dependence of the measurements.

\begin{table}[htbp]
  \begin{center}
    \caption{Average values of the number of
      participating nucleons (\npart) and of the nuclear overlap function
      (\taa) for the centrality bins used in this
      analysis.}
%    \vspace{1em}
    \label{tab:glauber}
    \begin{tabular}{rcccccc}
      \hline\vspace{0.1em}
%      ~ & \multicolumn{2}{c}{$\npart$} & \multicolumn{2}{c}{$\taa$ (mb$^{-1}$)}\\
      Centrality (\%) & \npart & \taa \\\hline
      0--5	& 381  & 25.9 $\pm$ 1.1 \\
      5--10	& 329  & 20.5 $\pm$ 0.9 \\
      10--20	& 261 & 14.5 $\pm$ 0.76 \\
      20--30	& 187 & 8.78 $\pm$ 0.58 \\
      30--40	& 130 & 5.09 $\pm$ 0.43 \\
      40--50	& 86.3 & 2.75 $\pm$ 0.30 \\ \hline
      50--70         & 42.0 & 0.985 $\pm$ 0.145 \\
      70--100       & 8.75 & 0.130 $\pm$ 0.020 \\ \hline
      50--100	& 22.1 & 0.486 $\pm$ 0.073 \\ \hline
      0--100	& 113 & 5.66 $\pm$ 0.35 \\ \hline
    \end{tabular}
  \end{center}
\end{table}

\subsection{Muon selection}

Muons are reconstructed using a global fit to a track in the muon detectors matched to a track in the silicon tracker. The offline muon reconstruction algorithm used for the PbPb data has been significantly improved relative to that used for the previous measurement~\cite{Chatrchyan:2012lxa}. The efficiency has been increased by running multiple iterations in the pattern recognition step, raising the number of reconstructed \PgUa\ by approximately 30\%. Background muons from cosmic rays and heavy-quark semileptonic decays are rejected by requiring a set of selection criteria on each muon track. The criteria used are based on previous studies of the performance of the muon reconstruction~\cite{Chatrchyan:2012xi}. At least one muon detector hit is required to be included in the global-muon track fit, and segments in at least two muon detectors are required to be matched to the track in the silicon tracker. To ensure a good \pt measurement, more than 10 hits in the tracker are required, and the $\chi^2$ per number of degrees of freedom of the tracker (global-muon) track fit is required to be less than 4 (10). To further reject cosmic muons and muons from decays in flight, the track is required to have a hit from at least one pixel detector layer and a transverse (longitudinal) distance of closest approach of less than 2 (5) mm from the measured primary vertex position. The probability that the two muons originate from the same vertex is required to be better than 1\%.

The individual muon-\pt are limited to be above 4~\GeVc for the \PgUb\ and \PgUc\ analyses as in previous publications~\cite{Chatrchyan:2012np,Chatrchyan:2011pe,Chatrchyan:2012lxa}, while one of them is allowed to go down to 3.5~\GeVc for the \PgUa. This raise the \PgUa\ yield by $\approx 45$\%, and its significance by up to 50\% depending on \pt and $y$. The resulting invariant mass distributions are shown n Fig.~\ref{fig:mass} for pp and PbPb collisions.

\begin{figure}[hbtp]
\centering
\includegraphics[width=0.45\textwidth]{massFit_pp.pdf}
\includegraphics[width=0.45\textwidth]{massFit_PbPb.pdf}
\caption{Dimuon invariant-mass distributions from the pp (left) and PbPb (right) data at $\sqrtsnn = 2.76$~TeV, for muon pairs having one \pt greater than 4.0\GeVc and the other greater than 3.5~\GeVc. The solid (signal + background) and dashed (background only) lines show the result of fits described in the text. The bottom panels show the pull of the data-fit differences.}
\label{fig:mass} 
\end{figure}

\section{Analysis}

\subsection{Signal extraction}

To extract the \PgUa, \PgUb\ and \PgUc\ yields, unbinned maximum-likelihood
fits to the $\mu^+ \mu^-$ invariant mass spectrum are performed between 7 and 14\GeVcc. 

Their results are displayed as lines on Fig.~\ref{fig:mass}, for \pt, rapidity and centrality integrated case. The \PgU resonances are modelled by the sum of two Crystal Ball (CB) functions, that is a Gaussian resolution functions with the low side tail replaced with a power law describing final state radiation. This choice was suggested by simulation studies, as well as by high-statistics pp analyses at 7~TeV~\cite{Chatrchyan:2013yna}. Given the relatively low statistic of the 2.76~TeV samples, most of the parameters are fixed to values provided by simulations, then allowed to vary to compute the associated systematic uncertainties. Only the \PgUa\ mass, and the \PgUa, \PgUb\ and \PgUc\ yields are left free. The mass of the \PgUb\ and \PgUc\ are constrained to be equal to the fitted \PgUa\ mass multiplied by the mass ratio from the Particle Data Group~\cite{Agashe:2014kda}. The background is modelled by an exponential function multiplied by an error function describing the low-mass turn-on, with all four parameters left free. 

This fitting procedure results in $2588 \pm 82$ and $4977 \pm 87$ \PgUa\ in PbPb and pp respectively, with one muon of \pt greater than 4.0 \GeVc and the other greater than 3.5. With both muons above 4.0 \GeVc, this yields to $142 \pm 42$ and $1183 \pm 49$ \PgUb\ in PbPb and pp respectively, and no ($12\pm 39$) and $601 \pm 41$ \PgUc\ in PbPb and pp respectively. 

\subsection{Acceptance and efficiency}

In order to correct yields for the acceptance and efficiency in the PbPb analysis, the three \PgU\ states have been simulated and embedded in PbPb events, with the same settings as in~\cite{Chatrchyan:2012lxa}. The acceptance, defined as the fraction of \PgUa\ mesons in the $y < |2.4|$ rapidity range that decay in muons of pseudorapidity $\eta < 2.4$ and \pt greater than 3.5 and 4.0 \GeVc, amounts to 35\%, with a drop for the mesons of the most forward rapidity or intermediate \pt. For \PgUb\ and \PgUc\, the acceptance is 28\% within the tighter single muon \pt selection. Within this acceptance, the reconstruction and trigger efficiency is 66 \%, 72 \% and 75 \% for the \PgUa, \PgUb and \PgUc\ respectively. 

To account for possible differences between data and simulations, the so-called tag-and-probe technique is used, as described in~\cite{Chatrchyan:2013yna}. Base on the corresponding \JPsi samples, single muon efficiencies are derived from both data and simulation. The small observed discrepancy are converted in \pt\- and $y$-dependent single muon correction factors that are applied to muons in the acceptance times efficiency simulation. The net correction to the \PgU\ yields ranges from 3 to 15\%, being stronger for low \pt or high rapidity. 

\subsection{Systematic uncertainties}

The systematic uncertainty on signal extraction is taken as the RMS of seven variations of the line shapes. Five corresponds to releasing each parameters of the signal line shape to allow for possible variations with respect to the simulated line shape. The other two consist in using more complicated shapes for the background: by adding a first or second order Chebychev polynomial functions to the default function. These result in uncertainty ranging from 0.3 to 6.6~\%, depending on the bin. 

The systematic uncertainty coming from the acceptance and efficiency is taken as 100\% of the data-simulation (tag-and-probe) correction, ranging from 3 to 15\%. 

\section{Results} \label{sec:results}

\subsection{Cross sections}

Figures~\ref{fig:CrossSection_ppAA_pt} and \ref{fig:CrossSection_ppAA_rap} shows the differential cross sections as functions of \pt and rapidity, respectively. Measured yields are corrected for acceptance and efficiency, then divided by the bin width in consideration, and normalized by 1) the measured luminosity in pp collisions (left figures), 2) the number of corresponding minimum bias events times the minimum bias thickness function for the PbPb collisions (right figures), putting the two of them on a comparable scale. In pp collisions, the data authorized measuring the three states with the same binning. In PbPb collisions, the same binning can be used for the \PgUa, but wider bins are necessary for the \PgUb. 

\begin{figure}[hbtp]
\centering
\includegraphics[width=0.45\textwidth]{CS1S_ppPt.pdf}
\includegraphics[width=0.45\textwidth]{CS_AAPt.pdf}
\caption{Differential cross section of \PgU\ states as a function of
 their transverse momentum in pp (right) and PbPb (left) collisions. From top to bottom, squares, circles and triangles stand for \PgUa, \PgUb\ and \PgUc, respectively. Statistical (systematic) uncertainties are displayed as error bars (boxes). }
\label{fig:CrossSection_ppAA_pt} 
\end{figure}

\begin{figure}[hbtp]
\centering
\includegraphics[width=0.45\textwidth]{CS1S_ppRap.pdf}
\includegraphics[width=0.45\textwidth]{CS_AA_Rap.pdf}
\caption{Differential cross section of \PgU\ states as a function of
  their rapidity in pp (right) and PbPb (left) collisions. From top to bottom, squares, circles and triangles stand for \PgUa, \PgUb\ and \PgUc, respectively. Statistical (systematic) uncertainties are displayed as error bars (boxes).}
\label{fig:CrossSection_ppAA_rap} 
\end{figure}

\subsection{Nuclear modification factors}

From the pp and PbPb yields and \taa values, nuclear modification factors, \raa, are derived. They are shown one Fig. \ref{fig:Raa_kin} as a function of the \PgU\ transverse momentum (left) and rapidity (right), showing a suppression of a factor of $\approx 2$ and  
10 for \PgUa and \PgUb, respectively. No pronounced dependence on the meson kinematics is observed, the values being constant within uncertainties as a function of both \pt and $y$. 

\begin{figure}[hbtp]
\centering
\includegraphics[width=0.4\textwidth]{RAA_Pt.pdf}
\includegraphics[width=0.4\textwidth]{RAA_Rap.pdf}
\caption{Nuclear modification factor of \PgUa and \PgUb\ in PbPb collisions as a
  function of transverse momentum (left) and rapidity (right). Statistical (systematic) uncertainties are displayed as error bars (boxes), while the global fully-correlated uncertainty is displayed as a box at unity. }
\label{fig:Raa_kin} 
\end{figure}

Figure~\label{fig:Raa_cent} shows \raa as a function of centrality, displayed as the number of participating nucleons, \npart. The strong centrality dependence, already observed in~\cite{Chatrchyan:2012lxa} is mapped out with more precision. 

\begin{figure}[hbtp]
\centering
\includegraphics[width=0.6\textwidth]{RAA_nPart.pdf}
\caption{Nuclear modification factor of \PgUa and \PgUb\ in PbPb as a
  function of centralty, displayed as the number of participating nucleons. Statistical (systematic) uncertainties are displayed as error bars (boxes), while the global fully-correlated uncertainty is displayed as a box at unity.}
\label{fig:Raa_cent} 
\end{figure}

The \PgUc\ being not observed, an \raa upper limit is derived, using the Feldman-Cousins prescription. Its result, together with the integrated \raa values of the \PgUa and \PgUb\ are: 
\begin{eqnarray}
\raa (\PgUa) & = & 0.470 \pm 0.018 \pm 0.062, \\ 
\raa (\PgUb) & = & 0.103 \pm 0.031  \pm 0.013, \\
\raa (\PgUc) & < & 0.14 {\rm \; at \; 95\% \; CL.} 
\end{eqnarray}

\section{Summary}

The \PgUa, \PgUb\ and \PgUc\ mesons have been searched for in PbPb collisions, and compared to their yields in pp collisions at the same energy, 2.76 TeV. The \PgUa and \PgUb\ are suppressed by a factor of $\approx 2$ and 10, respectively, while the unobserved \PgUc\ corresponds to a suppression by a factor of 7.1, at 95\% confidence level. Though a strong centrality dependence of the suppression is observed for the \PgUa and \PgUb\ as a function of centrality, no noticeable dependence is observed, neither as a function of transverse momentum, nor as a function of rapidity. 

% \section{CMS papers}

% There are currently three kinds of CMS papers supported by this system in addition to tdrs: ``CMS Analysis Note,''
%  ``CMS Physics Analysis Summary," and ``CMS Paper."
% The processing for these differs only in the header of the first page,
% which includes a different PDF figure  for each kind. The
% appropriate header is chosen by the switch used in the
% \texttt{tdr}
% command.

% This document only deals with papers set with Pdf\LaTeX. We found Pdf\LaTeX\ plus \texttt{cvs} to be a reliable system for the production of
% large documents such as the Physics TDRs and felt it would be useful to extend it to the production of shorter documents such as CMS Notes. As of 2010 \texttt{cvs} has been replaced by subversion (\texttt{svn}).

% \subsection{The mechanics of generating and typesetting papers}

% To start
% you will need to request a note directory in the \texttt{svn} repository from the TDR manager (currently \href{mailto:George.Alverson@cern.ch}{George Alverson}
% or \href{Lucas.Taylor@cern.ch}{Lucas Taylor}). It is best to supply a list of the lxplus usernames of the co-authors who are to have write access to the repository at the time of the request.

% To generate output, check out your note
% directory from \texttt{svn} following the example below. The \texttt{tag} below is the identifier for your paper, typically of the form XXX-YY-NNN.
% Following the sequence below will populate your local copy of the repository  with only your note and not include the other notes. If you have a note, use ``note".
% For a paper, use ``paper." [Notes: (1)~When running without Kerberos authentication, use svn+ssh://username@svn.cern.ch. (2)~At Fermilab, even using \texttt{kinit user@CERN.CH} is not sufficient without specifying a specific svn server node (i.e., 137.138.229.205) instead of svn.cern.ch.]

% \begin{verbatim}
% > svn co -N svn+ssh://svn.cern.ch/reps/tdr2 myDir
% > cd myDir
% > svn update utils
% > svn update -N [papers|notes]
% > svn update [papers|notes]/XXX-YY-NNN
% # use the following line for tcsh...
% # ..use -sh for bash:
% > eval `[papers|notes]/tdr runtime -csh`
% > cd [papers|notes]/XXX-YY-NNN/trunk
% # (edit the template, then to build the document)
% > tdr  --style=[paper|pas|an|note] b XXX-YY-NNN
% \end{verbatim}

% The \texttt{nodraft} switch is required to turn off the ``Draft" overlay text.

% If you wish to export your paper (for local work or for security), you can produce a tarball with all the necessary files with
% \begin{verbatim}
% > tdr --style=note --export b mynote.
% \end{verbatim}
% This will function on  Unix or Windows systems which have recent copies of \LaTeX\ (including \AmS-\LaTeX) and \texttt{perl} installed. We currently use the \texttt{sectsty, subfig,
% fancyhdr, mathpazo, rotating, fancybox, lineno, longtable, ifthen} and \texttt{natbib} styles, which may not be included in the default distribution, plus our own versions of \texttt{pdfdraftcopy} and the \texttt{pennames} particle name macros. The latter has been modified for use with the fonts required by our standard style and also to provide for automatic switching to an italic version when necessary.

% \section{\texttt{svn} commands}
% \texttt{svn} is similar in many ways to \texttt{cvs}. Once a repository has been checked out, the workflow is
% almost identical except for tagging. In \texttt{svn}, tagging is done by creating a new directory branch using
% the \texttt{svn copy} command. Please see the \texttt{svn} \href{http://svnbook.red-bean.com/en/1.5/svn-book.pdf}{manual} for details, particularly
% the chapter on \href{http://svnbook.red-bean.com/en/1.5/svn-book.pdf#svn.branchmerge}{branching and tagging} and \href{http://svnbook.red-bean.com/en/1.5/svn-book.pdf#svn.forcvs}{svn for cvs users}. Please do not change the depth of the directory structure to the top-level \TeX\ file for your document.

% \emph{Please make sure to configure your svn client:} edit \texttt{\~/.subversion/config} so that it appropriately tags pdf and other commonly used file types.
% \begin{verbatim}
% [auto-props]
% *.pdf = svn:mime-type=application/pdf
% *.png = svn:mime-type=image/png
% *.jpg = svn:mime-type=image/jpeg
% *.tex = svn:eol-style=native
% *.eps = svn:mime-type=application/postscript
% \end{verbatim}
% There are other useful settings as well. For example, to stop \texttt{svn} from asking to commit backup files and object files, you can set the \texttt{global-ignores} flag:
% \begin{verbatim}
% [miscellany]
% global-ignores = *.o *.bak
% \end{verbatim}


% \section{Document layout}

% \subsection{Standard macros}
% \textit{Notes} will automatically include
% \texttt{ptdr-definitions.sty}, which provides definitions
% for many physics and CMS-related entities, \eg, \GeVcc. These are discussed in more detail in section~\ref{sec:CMSmacros}, and a complete list is
% given the Appendix.%~\ref{app:symdef}.

% All style-related parameters are set in the
% class file included by the script and generally follow the article style. The chapter command is not implemented.

% \subsection{Title block}
% Please see the \LaTeX\ \href{https://svnweb.cern.ch/cern/wsvn/tdr2/papers/XXX-08-000/trunk/XXX-08-000.tex}{source} for this file to see how the title
% page is generated. In general it follows the normal \LaTeX\ practice for title pages.

% The type of note (PAS, AN, Note, etc.) is set
% through the \texttt{--style} switch in the \texttt{tdr} script. When in draft mode, the string ``Draft" is displayed on the page and the title block contains (in addition to the date), information about the svn status of the document and the PDF metadata.

% For ANs which need to differentiate between primary and non-primary authors, using the star form of the author macro will add a footnote to indicate a primary author: \linebreak\verb|\author*{A. Cern Person}|.

% \subsection{Page size, margins and fonts}

% The standard European  paper size is A4 (210\unit{mm} x 297\unit{mm} ($8.3''$ x
% $11.7''$)) while American paper is US Letter (216\unit{mm} x 279\unit{mm} ($8.5''$ x
% $11.0''$)), somewhat wider and shorter. In the era of straight
% PostScript this led to difficulties, but PDF print drivers now
% generally supply a  ``shrink and center" option. In this template
% we have set the \LaTeX\ page style parameters to match the standard A4 size
% (see Table~\ref{tab:page_layout}) and rely upon that option to
% produce an acceptable result on US Letter paper.

% Do not override the default fonts. They are currently set to be
% Palatino and Helvetica. The math fonts have also been changed to
% Palatino so that they do not clash with the body text,
% particularly in regards to numbers and units. This means the
% authors should use \verb|\text| commands to put text in subscripts
% and superscripts, and most importantly \emph{do not use}
% \verb|\rm| in formulas with Greek symbols, otherwise you will end up with formulae looking like the second one below.

% \begin{gather}
% \phi = \text{a Greek letter}\\
% {\rm \phi} = \text{a mistake}
% \end{gather}

% Also note that the math fonts include a full set of Greek symbols in Math Italic Bold (produced with \verb|\mathbold|),
% but only uppercase in Math Bold (\verb|\mathbf|). Use either \verb|\boldsymbol| or \verb|\boldmath| outside the math delimiters (\$) (but inside braces) to get bold symbols. Compare:

%  \begin{tabular}{ll}
% \verb|$\mathbold{\Psi \alpha}$|& $\mathbold{\Psi \alpha}$ \\
%  \verb|$\mathbf{\Psi \alpha}$|& $\mathbf{\Psi \alpha}$ \\
%  \verb|$\Psi \otimes \beta$|& $\Psi \otimes \beta$ \\
% %  \ifthenelse{\boolean{cms@external}}{}{\verb|{\boldmath{$\Psi \otimes \beta$}}|& {\boldmath{$\Psi \otimes \beta$}} \\}
%  \verb|$\boldsymbol{\Psi \otimes \beta}$|& $\boldsymbol{\Psi \otimes \beta}$.
%  \end{tabular}

% Note, however, that \verb|\mathbold| will not work for most journal styles.

% When Greek or symbol characters are used in the title, author, keywords or section heads, please use the \verb|\texorpdfstring|
% command to provide alternate versions. Acrobat cannot deal with \TeX\ characters and will ignore many of them for your PDF bookmark. See the following two subsections and check the corresponding bookmarks. (You may notice that this will produce four instances of ``Package hyperref Warning: Token not allowed in a PDFDocEncoded string" in the output log.)

% \subsection{\texorpdfstring{H$_\text{2}$O-$\alpha$}%
% {Water-alpha} Demo}
% The title for this subsection was set with \\
% \verb|\subsection{\texorpdfstring{H$_\text{2}$O-$\alpha$}|\-\verb|{Water-alpha}}|\\
% The use of \verb|\text| sets the numeral 2 in the same font and weight as the rest of the title (here Helvetica bold).

% \subsection{H$_2$O-$\alpha$ Demo}
% The title for this subsection was set with\\
% \verb|\subsection{H$_2$O-$\alpha$}|.

% \subsection{Tables, figures\label{sec:tab-fig}}

% Place the captions above the object for tables and use \texttt{topcaption}, below for figures using \texttt{caption}. To force a full width figure or table in the two-column mode of most journal reprint formats, use \verb|\textwidth| as the unit along with the starred versions of the commands:

% \begin{verbatim}
% \begin{figure*}[hbtp]\begin{center}
%  \includegraphics[width=0.95\textwidth]{CMS-bw-logo}
%  \caption{Figures inserted using includegraphics.}
%   \label{fig:ex1}\end{center}\end{figure*}
% \end{verbatim}

% \begin{table*}[htbH]
% \begin{center}
% \topcaption{An example table: Current page and paragraph layout
% parameters. ( 72.27\,pt = 1\,in )\label{tab:page_layout}}
% \begin{tabular}{lclc}
% \hline \verb|\hoffset| & \the\hoffset & \verb|\voffset| & \the\voffset \\
% \verb|\textheight| & \the\textheight & \verb|\textwidth| & \the\textwidth \\
% \verb|\baselineskip| & \the\baselineskip & \verb|\marginparsep| & \the\marginparsep \\
% \verb|\topmargin| & \the\topmargin &&\\
% \verb|\headheight| & \the\headheight & \verb|\footskip| & \the\footskip \\
% \verb|\oddsidemargin| & \the\oddsidemargin & \verb|\evensidemargin| & \the\evensidemargin \\
% \verb|\columnwidth| & \the\columnwidth&\verb|\linewidth|&\the\linewidth\\
% \hline
% \end{tabular}
% \end{center}
% \end{table*}


% Figures can include PDF files using the
% \texttt{includegraphics} package, which is automatically installed
% by our class file. A nice feature is that if a file extension is not supplied, \texttt{includegraphics} supplies
% an appropriate one based on whether the file is being Pdf\LaTeX{}ed or just \LaTeX{}ed. The package also can accept
% sizes to which the figures will be scaled. Specifying both width and height forces both
% dimensions to be changed and causes a distortion of the figure, however, so
% only use one of the two. Don't try to use scaling to correct a bad original aspect ratio. If neither width nor height is given, the size is
% taken from the Crop Box size embedded in the file, which is similar to the
% BoundingBox in PostScript. If there is too much white space around your figure, it may be that
% the Crop Box has been mis-set during a conversion from PostScript to PDF. Recommended
% translators on lxplus are \texttt{epstopdf} and \texttt{ps2pdf -dEPSCrop}. Native PostScript can not be included.

% The \texttt{subfig} package is included and can be used for PASs and ANs (but not papers) to generate (a), (b), \etc labels under the subfigures through the use of the \texttt{subfloat} command. We have aliased \texttt{subfigure} to \texttt{subfloat} to avoid breaking older documents which may have depended on the \texttt{subfigure} package, but the spacing will not necessarily be the same. You may need to add line breaks by hand.


% %This figure will be side-by-side in CMS mode and stacked for two-column journal layout. The labels track.
% %If the figures are of equal width and fill the page across the full width (width= ~0.45 \textwidth each) the use of \cmsFigWidth is not necessary.
% \begin{figure}[hbtp]
%   \begin{center}
%     \includegraphics[width=\cmsFigWidth]{CMS-bw-logo}
%     \includegraphics[width=\cmsFigWidth]{CMScol}
%  %   to generate (a) and (b) labels under the figures, you can use subfloat, but this is not recommended: takes too much space
%  %   \subfloat[]{\includegraphics[width=0.2\textwidth]{CMS-bw-logo}}\subfloat[]{\includegraphics[width=0.2\textwidth]{CMScol}}
%     \caption{Figures inserted using includegraphics. (\cmsLeft) Black and white. (\cmsRight) Color.}
%     \label{fig:ex1}
%   \end{center}
% \end{figure}

% When including root-generated figures, please make sure to use the standard macro to set the figure parameters,
% and to first generate the output in eps format which is then converted to PDF. The macro for TDR styles, \texttt{tdrstyle.C}, is available in the \texttt{utils/general} directory.

% The non-vector file types png and jpg are also picked up if present. Vector graphics is preferred except in cases such as scatter plots with millions of points. A screen grab saved as pdf is not vector graphics. In all cases, figures intended for publication should be publication quality.

% As a result of the file-tracking we use for export, please keep the length of the graphics files (including any subdirectory names and the period plus extension, which is not normally entered) shorter than 65 characters.

% %------------------------------------------------------------------------------
% \section{Standards\label{sec:standards}}
% Please check the
% \textit{\href{https://twiki.cern.ch/twiki/bin/view/CMS/Internal/PubGuidelines}{CMS
% Guidelines for Authors}} and the
% \textit{\href{https://svnweb.cern.ch/cern/wsvn/tdr2/utils/trunk/general/notes_for_authors.pdf}{Notes
% for TDR authors}} for authoritative information on CMS standards
% for publications and for tips on writing in \LaTeX. (If you find any discrepancies between those
% documents and the practices in this example, please contact
% \href{mailto:George.Alverson@cern.ch}{us}.)

% \subsection{Math}
% \textit{Notes} include the \AmS-\LaTeX\ class file which defines
% many additional math symbols, including \verb|\gtrsim|
% ($\gtrsim$). It also allows for better control in setting
% equations. Please see the \AmS-\LaTeX\
% \href{ftp://ftp.ams.org/pub/tex/doc/amsmath/amsldoc.pdf}{user
% guide} for complete details.

% As previously mentioned, uniformity of symbol use should be enforced through  use of the definitions  in \texttt{ptdr-definitions}.\\

% \subsection{Figure Style}
% Figures must have legible axis labels and values, symbol names, and line types when read at the final design size. For tdr-style documents, this is enforced through the use of the root macro file, tdrstyle.C, as discussed in Section~\ref{sec:tab-fig}.

% \subsection{Particle Names: \texorpdfstring{\PZz}{Z-0} to \texorpdfstring{\PJgy}{J/psi}}
% Most standard particle names can be typeset using the  the \texttt{pennames-pazo} package, which is an implementation of the PENNAMES (Particle Entity Names) scheme adapted by us for use with Palatino/mathpazo fonts, as far as possible. The advantage is that the formatting will mostly adhere to particle naming conventions for typesetting (no, particle names are not mathematical symbols--- they're more like units).

% \subsection{CMS Macros\label{sec:CMSmacros}}
% Macros introduced by CMS are listed in Appendix~\ref{app:symdef}. The macros for units are particularly useful, especially as the include the proper spacing between the magnitude and the unit (a thinspace), and they have an xspace at the end, which removes the necessity of ending them with a pair of braces. Thus, use \verb|a momentum of 5\TeVc was measured| to produce ``a momentum of 5\TeVc was measured."

% \section{Submitting a note}

% Please follow the rules and procedures defined on the
% \href{http://cms.cern.ch/iCMS/jsp/iCMS.jsp?mode=single&part=publications}{iCMS
% server} or on the CMS wiki page for analysis notes and other CMS note types. For PAS documents or papers intended for journals, the \href{http://cms.cern.ch/iCMS/jsp/analysis/admin/analysismanagement.jsp}{CADI} analysis management page controls submission.

% \section{References example}

% References (\cite{CMS_AN_2006-027,CMS_NOTE_2006-106,Brandt:1997gi, Buchmuller:1986zs,PTDR2,CMS-PAS-JME-10-003,LEPewkfits,Bertram:2000br,Khachatryan:2010mp, RooStats,nu2}) should use standard
% BibTeX citations and be placed in a separate bib file. This is automatically included by the \verb+\bibliograph{auto_generated}+ command placed at the end of the note.
% We recommend the use of inspirehep.net (SPIRES)
% identifiers as reference keys, where possible. This allows the reference to be easily found on Spires using the \textit{find texkey}
% command. It also ensures uniqueness if the references are to be combined into a larger bib file later. Note, however, that Spires tends to classify all bibliographic entities as Articles. Entities such as arXiv postings do not have an associated journal, though, and should be entered in the bib file as Unpublished. See the bib file for this note for examples, including the correct use of hyperlinks (all references should be linked when possible). Some journal styles will lowercase the titles in references, so use curly braces (\verb|{}|) to escape proper names and the like. Don't escape the entire title gratuitously.
% Direct references (\eg, see Ref.~\citenum{LEPewkfits}), may use the \verb|\citenum| form of \verb|\cite|.

% % >> acknowledgments (for journal papers)
% % Please include the latest version from https://twiki.cern.ch/twiki/bin/viewauth/CMS/Internal/PubAcknow.
% %\begin{acknowledgments}...ack-text...\end{acknowledgments}
% %

% %% **DO NOT REMOVE BIBLIOGRAPHY**
 \bibliography{auto_generated}   % will be created by the tdr script.

% %% examples of appendices. **DO NOT PUT \end{document} at the end
% %\clearpage
% \appendix
% \section{PTDR Symbol Definitions\label{app:symdef}}
% If absolutely necessary, symbol definitions may
% be over-ridden using the \verb|\renewcommand| command. If you don't want to over-ride the default version of a command but provide it for use outside the normal tdr system, please use the \verb|\providecommand| command.

% %
% % Below is a multi-column format to display the standard PTDR symbols.
% % Feel free to delete the lines from here to the end and the pdefs.tex file once you start modifying the template. The actual definitions are
% % pulled in by the \input{ptdr-definitions} at the beginning.
% %
% {
% \small%
% \providecommand{\symexamp}[2]
% {\makebox[\columnwidth]{\makebox[0.160\textwidth][l]{#1:}
%                           \makebox[0.025\textwidth][l]{~}
%                           \parbox[t]{0.26\textwidth}{#2\vspace*{-2.5mm}\hfill}}\\}
% \setlength{\columnseprule}{0.2mm}
% \ifthenelse{\boolean{cms@external}}{}{\setlength{\multicolsep}{7mm}
% %
% \vspace*{-5mm}
% \begin{multicols}{2}
% %
% \raggedcolumns%
% }
% \noindent%
% \input{pdefs}\\
% \ifthenelse{\boolean{cms@external}}{}{\end{multicols}}
% \clearpage
% \section{Particle Symbols\label{pennames}}
% \ifthenelse{\boolean{cms@external}}{}{\vspace*{-5mm}
% \begin{multicols}{2}
% %
% \raggedcolumns%
% }
% %
% \noindent%
% \symexamp{PAz}{\PAz}
% \symexamp{PBm}{\PBm}
% \symexamp{PBpm}{\PBpm}
% \symexamp{PBp}{\PBp}
% \symexamp{PBz}{\PBz}
% \symexamp{PB}{\PB}
% \symexamp{PDiz}{\PDiz}
% \symexamp{PDm}{\PDm}
% \symexamp{PDpm}{\PDpm}
% \symexamp{PDp}{\PDp}
% \symexamp{PDstiiz}{\PDstiiz}
% \symexamp{PDstpm}{\PDstpm}
% \symexamp{PDstz}{\PDstz}
% \symexamp{PDz}{\PDz}
% \symexamp{PD}{\PD}
% \symexamp{PEz}{\PEz}
% \symexamp{PHpm}{\PHpm}
% \symexamp{PHz}{\PHz}
% \symexamp{PJgy}{\PJgy}
% \symexamp{PKeiii}{\PKeiii}
% \symexamp{PKgmiii}{\PKgmiii}
% \symexamp{PKia}{\PKia}
% \symexamp{PKii}{\PKii}
% \symexamp{PKi}{\PKi}
% \symexamp{PKm}{\PKm}
% \symexamp{PKpm}{\PKpm}
% \symexamp{PKp}{\PKp}
% \symexamp{PKsta}{\PKsta}
% \symexamp{PKstb}{\PKstb}
% \symexamp{PKstiii}{\PKstiii}
% \symexamp{PKstii}{\PKstii}
% \symexamp{PKstiv}{\PKstiv}
% \symexamp{PKstz}{\PKstz}
% \symexamp{PKst}{\PKst}
% \symexamp{PKzL}{\PKzL}
% \symexamp{PKzS}{\PKzS}
% \symexamp{PKzeiii}{\PKzeiii}
% \symexamp{PKzgmiii}{\PKzgmiii}
% \symexamp{PKz}{\PKz}
% \symexamp{PK}{\PK}
% \symexamp{PLpm}{\PLpm}
% \symexamp{PLz}{\PLz}
% \symexamp{PN}{\PN}
% \symexamp{PNa}{\PNa}
% \symexamp{PNb}{\PNb}
% \symexamp{PNc}{\PNc}
% \symexamp{PNd}{\PNd}
% \symexamp{PNe}{\PNe}
% \symexamp{PNf}{\PNf}
% \symexamp{PNg}{\PNg}
% \symexamp{PNh}{\PNh}
% \symexamp{PNi}{\PNi}
% \symexamp{PNj}{\PNj}
% \symexamp{PNk}{\PNk}
% \symexamp{PNl}{\PNl}
% \symexamp{PNm}{\PNm}
% \symexamp{PSHpm}{\PSHpm}
% \symexamp{PSHz}{\PSHz}
% \symexamp{PSWpm}{\PSWpm}
% \symexamp{PSZz}{\PSZz}
% \symexamp{PSe}{\PSe}
% \symexamp{PSgg}{\PSgg}
% \symexamp{PSgm}{\PSgm}
% \symexamp{PSgn}{\PSgn}
% \symexamp{PSgt}{\PSgt}
% \symexamp{PSgxpm}{\PSgxpm}
% \symexamp{PSgxz}{\PSgxz}
% \symexamp{PSg}{\PSg}
% \symexamp{PSq}{\PSq}
% \symexamp{PWR}{\PWR}
% \symexamp{PWm}{\PWm}
% \symexamp{PWpr}{\PWpr}
% \symexamp{PWp}{\PWp}
% \symexamp{PW}{\PW}
% \symexamp{PZLR}{\PZLR}
% \symexamp{PZgc}{\PZgc}
% \symexamp{PZge}{\PZge}
% \symexamp{PZgy}{\PZgy}
% \symexamp{PZi}{\PZi}
% \symexamp{PZz}{\PZz}
% \symexamp{PaBz}{\PaBz}
% \symexamp{PaB}{\PaB}
% \symexamp{PaDz}{\PaDz}
% \symexamp{PaD}{\PaD}
% \symexamp{PaKz}{\PaKz}
% \symexamp{PaSq}{\PaSq}
% \symexamp{PagL}{\PagL}
% \symexamp{PagOp}{\PagOp}
% \symexamp{PagSm}{\PagSm}
% \symexamp{PagSp}{\PagSp}
% \symexamp{PagSz}{\PagSz}
% \symexamp{PagXp}{\PagXp}
% \symexamp{PagXz}{\PagXz}
% \symexamp{Pagne}{\Pagne}
% \symexamp{Pagngm}{\Pagngm}
% \symexamp{Pagngt}{\Pagngt}
% \symexamp{Paii}{\Paii}
% \symexamp{Pai}{\Pai}
% \symexamp{Pap}{\Pap}
% \symexamp{Paqb}{\Paqb}
% \symexamp{Paqc}{\Paqc}
% \symexamp{Paqd}{\Paqd}
% \symexamp{Paqs}{\Paqs}
% \symexamp{Paqt}{\Paqt}
% \symexamp{Paqu}{\Paqu}
% \symexamp{Paq}{\Paq}
% \symexamp{Paz}{\Paz}
% \symexamp{Pbgcia}{\Pbgcia}
% \symexamp{Pbgciia}{\Pbgciia}
% \symexamp{Pbgcii}{\Pbgcii}
% \symexamp{Pbgci}{\Pbgci}
% \symexamp{Pbgcza}{\Pbgcza}
% \symexamp{Pbgcz}{\Pbgcz}
% \symexamp{Pbi}{\Pbi}
% \symexamp{PcgLp}{\PcgLp}
% \symexamp{PcgS}{\PcgS}
% \symexamp{PcgXp}{\PcgXp}
% \symexamp{PcgXz}{\PcgXz}
% \symexamp{Pcgcii}{\Pcgcii}
% \symexamp{Pcgci}{\Pcgci}
% \symexamp{Pcgcz}{\Pcgcz}
% \symexamp{Pcgh}{\Pcgh}
% \symexamp{Pem}{\Pem}
% \symexamp{Pep}{\Pep}
% \symexamp{Pe}{\Pe}
% \symexamp{Pfia}{\Pfia}
% \symexamp{Pfib}{\Pfib}
% \symexamp{Pfiia}{\Pfiia}
% \symexamp{Pfiib}{\Pfiib}
% \symexamp{Pfiic}{\Pfiic}
% \symexamp{Pfiid}{\Pfiid}
% \symexamp{Pfiipr}{\Pfiipr}
% \symexamp{Pfii}{\Pfii}
% \symexamp{Pfiv}{\Pfiv}
% \symexamp{Pfi}{\Pfi}
% \symexamp{Pfza}{\Pfza}
% \symexamp{Pfzb}{\Pfzb}
% \symexamp{Pfz}{\Pfz}
% \symexamp{PgD}{\PgD}
% \symexamp{PgDa}{\PgDa}
% \symexamp{PgDb}{\PgDb}
% \symexamp{PgDc}{\PgDc}
% \symexamp{PgDd}{\PgDd}
% \symexamp{PgDe}{\PgDe}
% \symexamp{PgDf}{\PgDf}
% \symexamp{PgDh}{\PgDh}
% \symexamp{PgDi}{\PgDi}
% \symexamp{PgDj}{\PgDj}
% \symexamp{PgDk}{\PgDk}
% \symexamp{PgL}{\PgL}
% \symexamp{PgLa}{\PgLa}
% \symexamp{PgLb}{\PgLb}
% \symexamp{PgLc}{\PgLc}
% \symexamp{PgLd}{\PgLd}
% \symexamp{PgLe}{\PgLe}
% \symexamp{PgLf}{\PgLf}
% \symexamp{PgLg}{\PgLg}
% \symexamp{PgLh}{\PgLh}
% \symexamp{PgLi}{\PgLi}
% \symexamp{PgLj}{\PgLj}
% \symexamp{PgLk}{\PgLk}
% \symexamp{PgLl}{\PgLl}
% \symexamp{PgLm}{\PgLm}
% \symexamp{PgO}{\PgO}
% \symexamp{PgOm}{\PgOm}
% \symexamp{PgOma}{\PgOma}
% \symexamp{PgS}{\PgS}
% \symexamp{PgSa}{\PgSa}
% \symexamp{PgSb}{\PgSb}
% \symexamp{PgSc}{\PgSc}
% \symexamp{PgSd}{\PgSd}
% \symexamp{PgSe}{\PgSe}
% \symexamp{PgSf}{\PgSf}
% \symexamp{PgSg}{\PgSg}
% \symexamp{PgSh}{\PgSh}
% \symexamp{PgSi}{\PgSi}
% \symexamp{PgSm}{\PgSm}
% \symexamp{PgSp}{\PgSp}
% \symexamp{PgSz}{\PgSz}
% \symexamp{PgU}{\PgU}
% \symexamp{PgUa}{\PgUa}
% \symexamp{PgUb}{\PgUb}
% \symexamp{PgUc}{\PgUc}
% \symexamp{PgUd}{\PgUd}
% \symexamp{PgUe}{\PgUe}
% \symexamp{PgUf}{\PgUf}
% \symexamp{PgX}{\PgX}
% \symexamp{PgXa}{\PgXa}
% \symexamp{PgXb}{\PgXb}
% \symexamp{PgXc}{\PgXc}
% \symexamp{PgXd}{\PgXd}
% \symexamp{PgXe}{\PgXe}
% \symexamp{PgXm}{\PgXm}
% \symexamp{PgXz}{\PgXz}
% \symexamp{Pgfa}{\Pgfa}
% \symexamp{Pgfiii}{\Pgfiii}
% \symexamp{Pgf}{\Pgf}
% \symexamp{Pgg}{\Pgg}
% \symexamp{Pgha}{\Pgha}
% \symexamp{Pghb}{\Pghb}
% \symexamp{Pghpr}{\Pghpr}
% \symexamp{Pgh}{\Pgh}
% \symexamp{Pgmm}{\Pgmm}
% \symexamp{Pgmp}{\Pgmp}
% \symexamp{Pgm}{\Pgm}
% \symexamp{Pgne}{\Pgne}
% \symexamp{Pgngm}{\Pgngm}
% \symexamp{Pgngt}{\Pgngt}
% \symexamp{Pgoa}{\Pgoa}
% \symexamp{Pgob}{\Pgob}
% \symexamp{Pgoiii}{\Pgoiii}
% \symexamp{Pgo}{\Pgo}
% \symexamp{Pgpa}{\Pgpa}
% \symexamp{Pgpii}{\Pgpii}
% \symexamp{Pgpm}{\Pgpm}
% \symexamp{Pgppm}{\Pgppm}
% \symexamp{Pgpp}{\Pgpp}
% \symexamp{Pgpz}{\Pgpz}
% \symexamp{Pgp}{\Pgp}
% \symexamp{Pgra}{\Pgra}
% \symexamp{Pgrb}{\Pgrb}
% \symexamp{Pgriii}{\Pgriii}
% \symexamp{Pgr}{\Pgr}
% \symexamp{Pgt}{\Pgt}
% \symexamp{Pgya}{\Pgya}
% \symexamp{Pgyb}{\Pgyb}
% \symexamp{Pgyc}{\Pgyc}
% \symexamp{Pgyd}{\Pgyd}
% \symexamp{Pgy}{\Pgy}
% \symexamp{Phia}{\Phia}
% \symexamp{Pn}{\Pn}
% \symexamp{Pp}{\Pp}
% \symexamp{Pqb}{\Pqb}
% \symexamp{Pqc}{\Pqc}
% \symexamp{Pqd}{\Pqd}
% \symexamp{Pqs}{\Pqs}
% \symexamp{Pqt}{\Pqt}
% \symexamp{Pqu}{\Pqu}
% \symexamp{Pq}{\Pq}
% \symexamp{PsDipm}{\PsDipm}
% \symexamp{PsDm}{\PsDm}
% \symexamp{PsDp}{\PsDp}
% \symexamp{PsDst}{\PsDst}
% \noindent%
% \textcolor{gray}{\rule{\columnwidth}{0.2pt}}\\
% \noindent%
% Future PENNAMES\hspace{\fill}{\tiny{include \verb|\xspace|}}\\
% \symexamp{PH}{\PH}
% \symexamp{PJGy}{\PJGy}
% \symexamp{PBzs}{\PBzs}
% \symexamp{Pg}{\Pg}
% \symexamp{PSg}{\PSg}
% \symexamp{PSQ}{\PSQ}
% \symexamp{PXXG}{\PXXG}
% \symexamp{PXXSG}{\PXXSG}
% \symexamp{PSGcp}{\PSGcp}
% \symexamp{PSGc}{\PSGc}
% \symexamp{PSGcz}{\PSGcz}
% \symexamp{PSGczDo}{\PSGczDo}
% \symexamp{PSGczDt}{\PSGczDt}
% \symexamp{PSGcpm}{\PSGcpm}
% \symexamp{Pl}{\Pl}
% \symexamp{PAl}{\PAl}
% \symexamp{PGnl}{\PGnl}
% \symexamp{PAGnl}{\PAGnl}
% \symexamp{PQtpr}{\PQtpr}
% \symexamp{PAQtpr}{\PAQtpr}
% \symexamp{PQbpr}{\PQbpr}
% \symexamp{PAQbpr}{\PAQbpr}
% \symexamp{PGg}{\PGg}
% \symexamp{PKzS}{\PKzS}
% \symexamp{PBs}{\PBs}
% \symexamp{PSQt}{\PSQt}
% \symexamp{PZpr}{\PZpr}
% \symexamp{PGn}{\PGn}
% \symexamp{PAGn}{\PAGn}
% \ifthenelse{\boolean{cms@external}}{}{\end{multicols}}
% }
% \section{OS X specific instructions}
% These instructions are based on a clean installation of Mac OS X 10.7.3 (Lion). This release has current versions of both perl and svn.

% Download the TeXLive 2011 installation, \url{http://mirror.ctan.org/systems/mac/mactex/MacTeX.mpkg.zip}, and install (if not already done). This is a relatively large installation.

% If a simple \textit{kinit Your\_CERN\_Username@CERN.CH} doesn't allow you to access the svn repository in the standard fashion, you can follow the instructions at \url{http://svn.web.cern.ch/svn/howto.php#accessing-sshlinux} to set up an ssh key pair. I tried using the keychain, but it isn't supported in the included version of svn. There are commercial versions available with GUIs, and maybe even free versions---I didn't look very hard---but they are not necessary.

% Then follow the general instructions in \url{https://svnweb.cern.ch/cern/wsvn/tdr2/papers/XXX-08-000/trunk/XXX-08-000_temp.pdf} (this document) and \url{https://svnweb.cern.ch/cern/wsvn/tdr2/utils/trunk/general/notes_for_authors.pdf}.

% Additional style files are required to generate documents in the journal formats, and many of these need to be installed individually.
%%% DO NOT ADD \end{document}!

