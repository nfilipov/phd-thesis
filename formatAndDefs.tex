%%%%%%%%%%%%% si latex récent
%% modifier biblatex
%% modifier slashbox en diagbox


% \usepackage{ifthen}
\usepackage{epigraph}
\setlength{\epigraphwidth}{0.5\textwidth}

\usepackage{amsmath,amssymb}             % AMS Math
\usepackage[greek,francais,english]{babel}
\selectlanguage{english}
%\DecimalMathComma % pour éviter l'espace après la virgule dans les nombres
% pour pdflatex
\usepackage[utf8]{inputenc}
\usepackage[T1]{fontenc}

% pour xelatex
% \usepackage{fontspec}
% \usepackage{xunicode}
% \usepackage{xltxtra}

\usepackage[left=1.3in,right=1.3in,top=1.1in,bottom=1.1in,includefoot,includehead,headheight=13.6pt]{geometry}
\renewcommand{\baselinestretch}{1.05}

% fusionner les lignes des tableaux
\usepackage{multirow}

\usepackage[running]{lineno}

\usepackage{slashed}

% bibtex

% \usepackage[backref=true,bibstyle=phys,citestyle=numeric-comp,backend=biber]{biblatex} % nécessite un biblatex récent
%\usepackage[backref=true]{biblatex}

% \usepackage[square,numbers,sectionbib]{natbib}
% \usepackage{chapterbib}

% My new commands
\newcommand{\afac}[1]{\noindent \textcolor{red}{{\small \sc #1}}}

\usepackage{CJKutf8}
\usepackage{xspace}
%units
\usepackage[squaren,Gray,mediumqspace,thinspace,textstyle]{SIunits}
% \usepackage[]{siunitx}
%\usepackage{physics}
\usepackage{xr}
\newcommand{\snn}{\ensuremath{\sqrt{s_{NN}}}}
\newcommand{\s}{\ensuremath{\sqrt{s}}}
\newcommand{\ket}[1]{\ensuremath{|#1\rangle}\xspace}
\newcommand{\bra}[1]{\ensuremath{\langle #1|}\xspace}
\newcommand{\greek}[1]{{\selectlanguage{greek}#1}}
\newcommand{\egamma}{E_{\gamma}}
\newcommand{\unitMass}{\ensuremath{\giga\electronvolt/c^2}}
\newcommand{\MeVc}{\ensuremath{\mega\electronvolt/c}\xspace}
\newcommand{\GeVc}{\ensuremath{\giga\electronvolt/c}\xspace}
\newcommand{\TeVc}{\ensuremath{\tera\electronvolt/c}\xspace}
\newcommand{\keV}{\kilo\electronvolt}
\newcommand{\MeV}{\mega\electronvolt}
\newcommand{\MeVmass}{\ensuremath{\mega\electronvolt/c^2}}
\newcommand{\GeV}{\giga\electronvolt}
\newcommand{\TeV}{\tera\electronvolt}
\newcommand{\fb}{\femto\barn}
\newcommand{\pb}{\pico\barn}
\newcommand{\invmub}{\micro\reciprocal\barn\xspace}
\newcommand{\invpb}{\pico\reciprocal\barn\xspace}
\newcommand{\invfb}{\femto\reciprocal\barn\xspace}
\newcommand{\invnb}{\nano\reciprocal\barn\xspace}
\newcommand{\mm}{\milli\meter}
\newcommand{\cm}{\centi\meter}
\newcommand{\ns}{\nano\second}
\newcommand{\lumi}{\ensuremath{\mathcal{L}}}
\newcommand{\lumiunit}{\centi\meter\rpsquared\usk\reciprocal\second}
\newcommand{\pomeron}{\mathbb{P}}
\newcommand{\xp}{x_\pomeron}
\newcommand{\eff}{\ensuremath{\varepsilon}}
\newcommand{\Ctnp}{C\ensuremath{^{\text{TNP}}}}
\newcommand{\acc}{\ensuremath{\alpha}}
\newcommand{\dkap}{\Delta\kappa^{\gamma}}
\newcommand{\kap}{\kappa^{\gamma}}
\newcommand{\lam}{\lambda^{\gamma}}
\newcommand{\aOw}{\ensuremath{a_0^W}}
\newcommand{\aOz}{\ensuremath{a_0^Z}}
\newcommand{\aCw}{\ensuremath{a_C^W}}
\newcommand{\aCz}{\ensuremath{a_C^Z}}
\newcommand{\aOwL}{\ensuremath{a_0^W/\Lambda^2}}
\newcommand{\aOzL}{\ensuremath{a_0^Z/\Lambda^2}}
\newcommand{\aCwL}{\ensuremath{a_C^W/\Lambda^2}}
\newcommand{\aCzL}{\ensuremath{a_C^Z/\Lambda^2}}
\newcommand{\twosidep}[1]{\stackrel{\leftrightarrow}{\partial^{#1}}}
\newcommand{\wwgamma}{WW\gamma}
\newcommand{\WWgg}{\ensuremath{WW\gamma\gamma}}
\newcommand{\ggWW}{\ensuremath{WW\gamma\gamma}}
\newcommand{\ZZgg}{\ensuremath{ZZ\gamma\gamma}}
\newcommand{\ggZZ}{\ensuremath{ZZ\gamma\gamma}}
\newcommand{\brehm}{brehmsstrahlung}
\newcommand{\Eslash}{\mbox{$\rm E \kern-0.6em\slash$}}
\newcommand{\etmiss}{\ensuremath{\slashed{E}_T}\xspace}
% \newcommand{\etmiss}{\ensuremath{\not\!\!E_T}\xspace}
\newcommand{\mtmin}{\ensuremath{M^{\text{min}}_{T}~(\ell,\etmiss)}}
\newcommand{\minmt}{\ensuremath{M^{\text{min}}_{T}~(\ell,\etmiss)}}
\newcommand{\sUE}{\ensuremath{sU\!E_T}}
\newcommand{\UE}{\ensuremath{U\!E_T}}
\newcommand{\vUE}{\ensuremath{\vec{U\!E}_T}}
\newcommand{\mtt}{\ensuremath{M_{T2}}}
\newcommand{\etmissscaled}{\ensuremath{\etmiss^{\text{Scaled}}}}
\newcommand{\metscaled}{\ensuremath{\etmiss^{\text{Scaled}}}}
\newcommand{\etmissspecial}{\ensuremath{\etmiss^{\text{Special}}}}
\newcommand{\metspecial}{\ensuremath{\etmiss^{\text{Special}}}}
\newcommand{\ttbar}{\ensuremath{t\bar{t}}}
\newcommand{\drll}{\ensuremath{\mathcal{R} (e^+,e^-)}}
\newcommand{\dphill}{\ensuremath{\Delta\phi (e^+,e^-)}}
\newcommand{\massmetll}{\ensuremath{M_T (\etmiss, e_1, e_2)}}
\newcommand{\dd}{\mathop{}\mathopen{}\text{d}}
% \newcommand{\Lcutoff}{\ensuremath{\Lambda_{\text{cutoff}}}}
\newcommand{\Lcutoff}{\ensuremath{\Lambda}}

% attention, avec xelatex, la commande Tr est déjà définie
\DeclareMathOperator{\tr}{Tr}

\newcommand{\gpU}[1]{\ensuremath{\text{U}({#1})}}
\newcommand{\gpSU}[1]{\ensuremath{\text{SU}({#1})}}
\newcommand{\gpO}[1]{\ensuremath{\text{O}({#1})}}
\newcommand{\gpSO}[1]{\ensuremath{\text{SO}({#1})}}

\newcommand{\DO}{D\O{}\xspace}
\newcommand{\FIXME}{\textcolor{red}{FIXME!!}}
\newcommand{\fixme}{\textcolor{red}{FIXME!!}}
\newcommand{\RunO}{Run~0}
\newcommand{\RunI}{Run~I}
\newcommand{\RunII}{Run~II}
\newcommand{\RunIIa}{Run~IIa}
\newcommand{\RunIIb}{Run~IIb}
% \newcommand{\RunIIb1}{Run~IIb1}
% \newcommand{\RunIIb2}{Run~IIb2}
% \newcommand{\RunIIb3}{Run~IIb3}
% \newcommand{\RunIIb4}{Run~IIb4}
\newcommand{\MS}{modèle standard\xspace}
\newcommand{\njet}{\ensuremath{N_{\textrm{jet}}}}
\newcommand{\minQual}{minQual\xspace}
\newcommand{\minEmv}{minEmv\xspace}
\newcommand{\emv}{EMV\xspace}
\newcommand{\lhood}{lhood8\xspace}
\newcommand{\dydt}{DY-DT\xspace}
\newcommand{\wwdt}{$WW$-DT\xspace}
\newcommand{\fddt}{Fd-DT\xspace}

\newcommand{\ith}{\textsuperscript{th}}
\newcommand{\st}{\textsuperscript{st}}
\newcommand{\nd}{\textsuperscript{nd}}
\newcommand{\rd}{\textsuperscript{rd}}

\newcommand{\tech}[1]{\emph{#1}}
% \newcommand{\cffig}[1]{{\it cf.} Fig.~\ref{#1}}
\newcommand{\cffig}[1]{voir figure~\ref{#1}}
\newcommand{\cffigg}[2]{voir figures~\ref{#1} et \ref{#2}}

% \myfig[width=3cm]{label}{images/Chap1/myfig.pdf}{The capture of my figure.}
\newcommand{\myfig}[4][width=0.7\textwidth]{%
 \begin{figure}[htbp]
   \begin{center}
     \includegraphics[#1]{#3}
  
     \caption{\label{#2} #4}
   \end{center}
 \end{figure}
}

% draw a spin (arrow and dot) with mfpic
\newcommand{\myspin}[3]{%args: xcenter, ycenter, angle
   \pointcolor{red}
   \drawcolor{green}
   \headcolor{green}
   \penwd{1.5pt}
   \begin{coords}
   \setlength{\headlen}{8pt}
   \headshape{.5}{2}{true}
   \shift{(#1,#2)}\rotate{#3}\arrow*\lines{(0,-7),(0,10)}
   \end{coords}
   \point[4pt]{(#1,#2)}
}

% % quotes
% % un boite pour l ’auteur de la citation
% \newsavebox{\auteurcitation}
% \newsavebox{\boitecitation}
% \newboolean{auteurcitationpresent}
% \newenvironment{myquote}[1][] {% argument optionnel vide par défaut
% % Clause begin :
% % on note si on a un auteur pour la citation ou pas
% \ifthenelse{\equal{#1}{}}{%
% \setboolean{auteurcitationpresent}{false}}{%
% \setboolean{auteurcitationpresent}{true}%
% \savebox{\auteurcitation}{#1}}%
% \begin{lrbox}{\boitecitation}
% \begin{minipage}{.8\linewidth}%
% \setlength{\parindent}{10pt}%
% \small\slshape \og \ignorspaces}% on passe en petit et penché
% {\unskip \fg
% % clause end de l ’environnement
% % s’ il y a un auteur on le met poussé tout à droite
% \ifthenelse{\boolean{auteurcitation}}%
% {\par\nopagebreak\hfill\usebox{\auteurcitation}}
% {}% sinon on ne fait rien ...
% \end{minipage}
% \end{lrbox}
% \begin{center}
% \usebox{\boitecitation}
% \end{center}}
\newcommand{\myepigraph}[2]{\epigraph{\slshape #1}{#2}}
\newcommand{\myenglishepigraph}[2]{\epigraph{\begin{otherlanguage}{english}\slshape
#1\end{otherlanguage}}{\begin{otherlanguage}{english}#2\end{otherlanguage}}}

%%%%%%%%%%%%%%%
% more defs
%%%%%%%%%%%%%%%
\newcommand{\eqdef}{\ensuremath{\stackrel{\tiny\textrm{def}}{=}}}
%LaTeX definitions
\newcommand{\CB}{{\rm CB}}
\newcommand{\Sig}{{\cal S}}
\newcommand{\signif}{\ensuremath {\mathcal{S}_c}}
\newcommand{\B}{{\cal B}}
\newcommand{\F}{{\cal F}}
%\newcommand{\NFitOneS}{{\cal N}_{\rm 1S}}
%\newcommand{\NFitTwoS}{{\cal N}_{\rm 2S}}
%\newcommand{\NFitThreeS}{{\cal N}_{\rm 3S}}
\newcommand{\NFitOneS}{\ensuremath{{\cal N}_{\rm 1S}}}
\newcommand{\NFit}{\ensuremath{{\cal N}_{\rm Fit}}}
\newcommand{\NFitTwoS}{\ensuremath{{\cal N}_{\rm 2S}}}
\newcommand{\NFitThreeS}{\ensuremath{{\cal N}_{\rm 3S}}}
\newcommand{\pt}{\ensuremath{p_{T}}\xspace}
\newcommand{\y}{\ensuremath{y}\xspace}
\newcommand{\NFitnS}{\ensuremath{{\cal N}_{\rm nS}}}
\newcommand{\NFitnSPbPb}{\ensuremath{{\cal N}_{\rm nS}^{\rm PbPb}}}
\newcommand{\NBkgd}{{\cal N}_{\rm bkgd}}
\newcommand{\TAA}{\ensuremath{T_{\rm AA}}}
\newcommand{\RAA}{\ensuremath{R_{\rm AA}}}
\newcommand{\NMB}{\ensuremath{N_{\rm MB}}}
\newcommand{\YPbPb}{\ensuremath{{\cal Y}_{\rm PbPb}}}
\newcommand{\Npart}{\ensuremath{N_{\rm part}}}
\newcommand{\Ncoll}{\ensuremath{N_{\rm coll}}}
\newcommand{\GEN}{\textrm{GEN}}
\newcommand{\RECO}{\textrm{RECO}}
\newcommand{\vs}{\textit{vs}}
\newcommand{\Jpsi}{\ensuremath{J\hspace{-.08em}/\hspace{-.14em}\psi}\xspace} % J/Psi (no mass)
%%%%%%%%%%%%%%%%%%%%%%%%%%%%%%%%%%%%%%%%%%%%%%%%%%%%%%%%%%%%%%%%%%%%%%
%                                                                    %
%  This is pennames.sty                                              %
%                                                                    %
%  It contains the definition of the short names for the PEN         %
%  Elementary Particle Naming Scheme, described in CNL 203, pp 8-11  %
%                                                                    %
%  Version 1.0: Original version -  4 Oct 1991 (evh)                 %
%          1.1: \def,\relax\ifmmode instead of \mbox                 %
%               16 Oct 1991 (mg)                                     %
%          1.2: Corrections for upsilon and psi - 21 Oct 1991 (evh)  %
%          1.3: Line lenghts < 80 charcaters - 22 Oct 1991 (mg)      %
%          1.4: Add definitions for NFSS (\mathrm instead of \rm)    %
%               27 May 1993 (mg)                                     %
%          1.5: Add definitions \PaD, \PaDz, \PaB, \PaBz             %
%               \Pq, \Paq, \Pqd, \Paqd, \Pqu, \Paqu, \Pqs, \Paqs     %
%               \Pqc, \Paqc, \Pqb, \Paqb, \Pqt, \Paqt, PaP, PagL     %
%               \PagSm, \PagSp, \PagSz, \PagXz, \PagXp, \PagOp, \PaSq%
%               12 Jul 1993 (mg)                                     %
%          1.6: Include \relax to force expansion of \if (DCa)       %
%               Add % at end of every command to eliminate possible  %
%               parasitic white space.                               %
%                6 Feb 1994 (mg)                                     %
%          2.0: Adapt to LaTeX2e with \ensuremath command            %
%                30 Jan 1995 (mg)                                    %
%          3.0: Make latex2e reference and define                    %
%               \newcommand and/or \ensuremath is undefined.         %
%                 3 Apr 1996 (mg)                                    %
%                                                                    %
%  Authors: Michel Goossens and Eric van Herwijnen                   %
%           CERN, Geneva, Switzerland                                %
%                                                                    %
%  Last Mod.  3 Apr 1996 (mg)                                        %
%                                                                    %
%  Adapted for use with Palatino/mathpazo (no roman for greek)       %
%  20 Jan 2010 (goa)                                                 %
%                                                                    %
%%%%%%%%%%%%%%%%%%%%%%%%%%%%%%%%%%%%%%%%%%%%%%%%%%%%%%%%%%%%%%%%%%%%%%
\newcommand{\PAz}{\ensuremath{\mathrm{A^0}}}
\newcommand{\PBm}{\ensuremath{{\mathrm{B}^{-}}}}
\newcommand{\PBpm}{\ensuremath{{\mathrm{B}^{\pm}}}}
\newcommand{\PBp}{\ensuremath{{\mathrm{B}^{+}}}}
\newcommand{\PBz}{\ensuremath{{\mathrm{B}^0}}}
\newcommand{\PB}{\ensuremath{{\mathrm{B}}}}
\newcommand{\PDiz}{\ensuremath{{\mathrm{D}_{1}(2420)^0}}}
\newcommand{\PDm}{\ensuremath{\mathrm{D^-}}}
\newcommand{\PDpm}{\ensuremath{\mathrm{D^{\pm}}}}
\newcommand{\PDp}{\ensuremath{\mathrm{D^+}}}
\newcommand{\PDstiiz}{\ensuremath{{\mathrm{D}^{\ast}_{2}(2460)^0}}}
\newcommand{\PDstpm}{\ensuremath{{\mathrm{D}^{\ast}(2010)^{\pm}}}}
\newcommand{\PDstz}{\ensuremath{{\mathrm{D}^{\ast}(2010)^0}}}
\newcommand{\PDz}{\ensuremath{\mathrm{D^0}}}
\newcommand{\PD}{\ensuremath{\mathrm{D}}}
\newcommand{\PEz}{\ensuremath{\mathrm{E^0}}}
\newcommand{\PHpm}{\ensuremath{\mathrm{H^{\pm}}}}
\newcommand{\PHz}{\ensuremath{\mathrm{H^0}}}
\newcommand{\PJgy}{\ensuremath{\mathrm{J}\hspace{-.08em}/\hspace{-.14em}\psi\mathrm{(1S)}}}
\newcommand{\PKeiii}{\ensuremath{\mathrm{K}_\mathrm{e3}}}
\newcommand{\PKgmiii}{\ensuremath{\mathrm{K}_{\mu \mathrm{3}}}}
\newcommand{\PKia}{\ensuremath{\mathrm{K_1(1400)}}}
\newcommand{\PKii}{\ensuremath{\mathrm{K_2(1770)}}}
\newcommand{\PKi}{\ensuremath{\mathrm{K_1(1270)}}}
\newcommand{\PKm}{\ensuremath{\mathrm{K^-}}}
\newcommand{\PKpm}{\ensuremath{\mathrm{K^{\pm}}}}
\newcommand{\PKp}{\ensuremath{\mathrm{K^+}}}
\newcommand{\PKsta}{\ensuremath{\mathrm{K^{\ast}(1370)}}}
\newcommand{\PKstb}{\ensuremath{\mathrm{K^{\ast}(1680)}}}
\newcommand{\PKstiii}{\ensuremath{\mathrm{K^{\ast}_3(1780)}}}
\newcommand{\PKstii}{\ensuremath{\mathrm{K^{\ast}_2(1430)}}}
\newcommand{\PKstiv}{\ensuremath{\mathrm{K^{\ast}_4(2045)}}}
\newcommand{\PKstz}{\ensuremath{\mathrm{K^{\ast}_0(1430)}}}
\newcommand{\PKst}{\ensuremath{\mathrm{K^{\ast}(892)}}}
\newcommand{\PKzL}{\ensuremath{\mathrm{K^0_L}}}
\newcommand{\PKzS}{\ensuremath{\mathrm{K^0_S}}}
\newcommand{\PKzeiii}{\ensuremath{\mathrm{K^0_{e3}}}}
\newcommand{\PKzgmiii}{\ensuremath{\mathrm{K^0_{\mu 3}}}}
\newcommand{\PKz}{\ensuremath{\mathrm{K^0}}}
\newcommand{\PK}{\ensuremath{\mathrm{K}}}
\newcommand{\PLpm}{\ensuremath{\mathrm{L^{\pm}}}}
\newcommand{\PLz}{\ensuremath{\mathrm{L^0}}}
\newcommand{\PN}{\ensuremath{\mathrm{N}}}
\newcommand{\PNa}{\ensuremath{\mathrm{N(1440)P_{11}}}}
\newcommand{\PNb}{\ensuremath{\mathrm{N(1520)D_{13}}}}
\newcommand{\PNc}{\ensuremath{\mathrm{N(1535)S_{11}}}}
\newcommand{\PNd}{\ensuremath{\mathrm{N(1650)S_{11}}}}
\newcommand{\PNe}{\ensuremath{\mathrm{N(1675)D_{15}}}}
\newcommand{\PNf}{\ensuremath{\mathrm{N(1680)F_{15}}}}
\newcommand{\PNg}{\ensuremath{\mathrm{N(1700)D_{13}}}}
\newcommand{\PNh}{\ensuremath{\mathrm{N(1710)P_{11}}}}
\newcommand{\PNi}{\ensuremath{\mathrm{N(1720)P_{13}}}}
\newcommand{\PNj}{\ensuremath{\mathrm{N(2190)G_{17}}}}
\newcommand{\PNk}{\ensuremath{\mathrm{N(2220)H_{19}}}}
\newcommand{\PNl}{\ensuremath{\mathrm{N(2250)G_{19}}}}
\newcommand{\PNm}{\ensuremath{\mathrm{N(2600)I_{1,11}}}}
\newcommand{\PSHpm}{\ensuremath{\mathrm{\widetilde{H}^{\pm_j}}}}
\newcommand{\PSHz}{\ensuremath{\mathrm{\widetilde{H}^0_j}}}
\newcommand{\PSWpm}{\ensuremath{\mathrm{\widetilde{W}^{\pm}}}}
\newcommand{\PSZz}{\ensuremath{\mathrm{\widetilde{Z}^0}}}
\newcommand{\PSe}{\ensuremath{\mathrm{\widetilde{e}}}}
\newcommand{\PSgg}{\ensuremath{{\widetilde{\gamma}}}}
\newcommand{\PSgm}{\ensuremath{{\widetilde{\mu}}}}
\newcommand{\PSgn}{\ensuremath{{\widetilde{\nu}}}}
\newcommand{\PSgt}{\ensuremath{{\widetilde{\tau}}}}
\newcommand{\PSgxpm}{\ensuremath{{\widetilde{\chi}^\pm_\mathrm{i}}}}
\newcommand{\PSgxz}{\ensuremath{{\widetilde{\chi}^0_\mathrm{i}}}}
\newcommand{\PSg}{\ensuremath{\mathrm{\widetilde{g}}}}
\newcommand{\PSq}{\ensuremath{\mathrm{\widetilde{q}}}}
\newcommand{\PWR}{\ensuremath{\mathrm{W_R}}}
\newcommand{\PWm}{\ensuremath{\mathrm{W^-}}}
\newcommand{\PWpr}{\ensuremath{\mathrm{W^{'}}}}%\prime
\newcommand{\PWp}{\ensuremath{\mathrm{W^+}}}
\newcommand{\PW}{\ensuremath{\mathrm{W}}}
\newcommand{\PZLR}{\ensuremath{\mathrm{Z_{LR}}}}
\newcommand{\PZgc}{\ensuremath{\mathrm{Z}_{\chi}}}
\newcommand{\PZge}{\ensuremath{\mathrm{Z}_{\eta}}}
\newcommand{\PZgy}{\ensuremath{\mathrm{Z}_{\psi}}}
\newcommand{\PZi}{\ensuremath{\mathrm{Z_1}}}
\newcommand{\PZz}{\ensuremath{\mathrm{Z^0}}}
\newcommand{\PaBz}{\ensuremath{\mathrm{\overline{B}}^0}}
\newcommand{\PaB}{\ensuremath{\mathrm{\overline{B}}}}
\newcommand{\PaDz}{\ensuremath{\mathrm{\overline{D}^0}}}
\newcommand{\PaD}{\ensuremath{\overline{\mathrm{D}}}}
\newcommand{\PaKz}{\ensuremath{\mathrm{\overline{K}^0}}}
\newcommand{\PaSq}{\ensuremath{\mathrm{\overline{\widetilde{q}}}}}
\newcommand{\PagL}{\ensuremath{{\overline{\Lambda}}}}
\newcommand{\PagOp}{\ensuremath{{\overline{\Omega}^+}}}
\newcommand{\PagSm}{\ensuremath{{\overline{\Sigma}^-}}}
\newcommand{\PagSp}{\ensuremath{{\overline{\Sigma}^+}}}
\newcommand{\PagSz}{\ensuremath{{\overline{\Sigma}^0}}}
\newcommand{\PagXp}{\ensuremath{{\overline{\Xi}^+}}}
\newcommand{\PagXz}{\ensuremath{{\Xi^0}}}
\newcommand{\Pagne}{\ensuremath{{\overline{\nu}_\mathrm{e}}}}
\newcommand{\Pagngm}{\ensuremath{{\overline{\nu}_{\mu}}}}
\newcommand{\Pagngt}{\ensuremath{{\overline{\nu}_{\tau}}}}
\newcommand{\Paii}{\ensuremath{\mathrm{a_2(1320)}}}
\newcommand{\Pai}{\ensuremath{\mathrm{a_1(1260)}}}
\newcommand{\Pap}{\ensuremath{\mathrm{\overline{p}}}}
\newcommand{\Paqb}{\ensuremath{\mathrm{\overline{q}_b}}}
\newcommand{\Paqc}{\ensuremath{\mathrm{\overline{q}_c}}}
\newcommand{\Paqd}{\ensuremath{\mathrm{\overline{q}_d}}}
\newcommand{\Paqs}{\ensuremath{\mathrm{\overline{q}_s}}}
\newcommand{\Paqt}{\ensuremath{\mathrm{\overline{q}_t}}}
\newcommand{\Paqu}{\ensuremath{\mathrm{\overline{q}_u}}}
\newcommand{\Paq}{\ensuremath{\mathrm{\overline{q}}}}
\newcommand{\Paz}{\ensuremath{\mathrm{a_0(980)}}}
\newcommand{\Pbgcia}{\ensuremath{\chi_\mathrm{b1}\mathrm{(2P)}}}
\newcommand{\Pbgciia}{\ensuremath{\chi_\mathrm{b2}\mathrm{(2P)}}}
\newcommand{\Pbgcii}{\ensuremath{\chi_\mathrm{b2}\mathrm{(1P)}}}
\newcommand{\Pbgci}{\ensuremath{\chi_\mathrm{b1}\mathrm{(1P)}}}
\newcommand{\Pbgcza}{\ensuremath{\chi_\mathrm{b0}\mathrm{(2P)}}}
\newcommand{\Pbgcz}{\ensuremath{\chi_\mathrm{b0}\mathrm{(1P)}}}
\newcommand{\Pbi}{\ensuremath{\mathrm{b_1(1235)}}}
\newcommand{\PcgLp}{\ensuremath{\Lambda_\mathrm{c}^+}}
\newcommand{\PcgS}{\ensuremath{\Sigma_\mathrm{c}\mathrm{(2455)}}}
\newcommand{\PcgXp}{\ensuremath{\Xi_\mathrm{c}^+}}
\newcommand{\PcgXz}{\ensuremath{\Xi_\mathrm{c}^0}}
\newcommand{\Pcgcii}{\ensuremath{\chi_\mathrm{c2}\mathrm{(1P)}}}
\newcommand{\Pcgci}{\ensuremath{\chi_\mathrm{c1}\mathrm{(1P)}}}
\newcommand{\Pcgcz}{\ensuremath{\chi_\mathrm{c0}\mathrm{(1P)}}}
\newcommand{\Pcgh}{\ensuremath{\eta_\mathrm{c}\mathrm{(1S)}}}
\newcommand{\Pem}{\ensuremath{\mathrm{e}^-}}
\newcommand{\Pep}{\ensuremath{\mathrm{e}^+}}
\newcommand{\Pe}{\ensuremath{\mathrm{e}}}
\newcommand{\Pfia}{\ensuremath{\mathrm{f}_1(1390)}}
\newcommand{\Pfib}{\ensuremath{\mathrm{f}_1(1510)}}
\newcommand{\Pfiia}{\ensuremath{\mathrm{f}_2(1720)}}
\newcommand{\Pfiib}{\ensuremath{\mathrm{f}_2(2010)}}
\newcommand{\Pfiic}{\ensuremath{\mathrm{f}_2(2300)}}
\newcommand{\Pfiid}{\ensuremath{\mathrm{f}_2(2340)}}
\newcommand{\Pfiipr}{\ensuremath{\mathrm{f}^{'}_2(1525)}}%\prime
\newcommand{\Pfii}{\ensuremath{\mathrm{f}_2(1270)}}
\newcommand{\Pfiv}{\ensuremath{\mathrm{f}_4(2050)}}
\newcommand{\Pfi}{\ensuremath{\mathrm{f}_1(1285)}}
\newcommand{\Pfza}{\ensuremath{\mathrm{f}_0(1400)}}
\newcommand{\Pfzb}{\ensuremath{\mathrm{f}_0(1590)}}
\newcommand{\Pfz}{\ensuremath{\mathrm{f}_0(975)}}
\newcommand{\PgD}{\ensuremath{\Delta}}
\newcommand{\PgDa}{\ensuremath{\Delta\mathrm{(1232)P_{33}}}}
\newcommand{\PgDb}{\ensuremath{\Delta\mathrm{(1620)S_{31}}}}
\newcommand{\PgDc}{\ensuremath{\Delta\mathrm{(1700)D_{33}}}}
\newcommand{\PgDd}{\ensuremath{\Delta\mathrm{(1900)S_{31}}}}
\newcommand{\PgDe}{\ensuremath{\Delta\mathrm{(1905)F_{35}}}}
\newcommand{\PgDf}{\ensuremath{\Delta\mathrm{(1910)P_{31}}}}
\newcommand{\PgDh}{\ensuremath{\Delta\mathrm{(1920)P_{33}}}}
\newcommand{\PgDi}{\ensuremath{\Delta\mathrm{(1930)D_{35}}}}
\newcommand{\PgDj}{\ensuremath{\Delta\mathrm{(1950)F_{37}}}}
\newcommand{\PgDk}{\ensuremath{\Delta\mathrm{(2420)H_{3,11}}}}
\newcommand{\PgL}{\ensuremath{\Lambda}}
\newcommand{\PgLa}{\ensuremath{\Lambda\mathrm{(1405) S_{01}}}}
\newcommand{\PgLb}{\ensuremath{\Lambda\mathrm{(1520) D_{03}}}}
\newcommand{\PgLc}{\ensuremath{\Lambda\mathrm{(1600) P_{01}}}}
\newcommand{\PgLd}{\ensuremath{\Lambda\mathrm{(1670) S_{01}}}}
\newcommand{\PgLe}{\ensuremath{\Lambda\mathrm{(1690) D_{03}}}}
\newcommand{\PgLf}{\ensuremath{\Lambda\mathrm{(1800) S_{01}}}}
\newcommand{\PgLg}{\ensuremath{\Lambda\mathrm{(1810) P_{01}}}}
\newcommand{\PgLh}{\ensuremath{\Lambda\mathrm{(1820) F_{05}}}}
\newcommand{\PgLi}{\ensuremath{\Lambda\mathrm{(1830) D_{05}}}}
\newcommand{\PgLj}{\ensuremath{\Lambda\mathrm{(1890) P_{03}}}}
\newcommand{\PgLk}{\ensuremath{\Lambda\mathrm{(2100) G_{07}}}}
\newcommand{\PgLl}{\ensuremath{\Lambda\mathrm{(2110) F_{05}}}}
\newcommand{\PgLm}{\ensuremath{\Lambda\mathrm{(2350) H_{09}}}}
\newcommand{\PgO}{\ensuremath{\Omega}}
\newcommand{\PgOm}{\ensuremath{\Omega^-}}
\newcommand{\PgOma}{\ensuremath{\Omega\mathrm{(2250)^-}}}
\newcommand{\PgS}{\ensuremath{\Sigma}}
\newcommand{\PgSa}{\ensuremath{\Sigma\mathrm{(1385) P_{13}}}}
\newcommand{\PgSb}{\ensuremath{\Sigma\mathrm{(1660) P_{11}}}}
\newcommand{\PgSc}{\ensuremath{\Sigma\mathrm{(1670) D_{13}}}}
\newcommand{\PgSd}{\ensuremath{\Sigma\mathrm{(1750) S_{11}}}}
\newcommand{\PgSe}{\ensuremath{\Sigma\mathrm{(1775) D_{15}}}}
\newcommand{\PgSf}{\ensuremath{\Sigma\mathrm{(1915) F_{15}}}}
\newcommand{\PgSg}{\ensuremath{\Sigma\mathrm{(1940) D_{13}}}}
\newcommand{\PgSh}{\ensuremath{\Sigma\mathrm{(2030) F_{17}}}}
\newcommand{\PgSi}{\ensuremath{\Sigma\mathrm{(2050)}}}
\newcommand{\PgSm}{\ensuremath{\Sigma^-}}
\newcommand{\PgSp}{\ensuremath{\Sigma^+}}
\newcommand{\PgSz}{\ensuremath{\Sigma^\mathrm{0}}}
\newcommand{\PgU}{\ensuremath{\Upsilon}}
\newcommand{\PgUa}{\ensuremath{\Upsilon\mathrm{(1S)}}}
\newcommand{\PgUb}{\ensuremath{\Upsilon\mathrm{(2S)}}}
\newcommand{\PgUc}{\ensuremath{\Upsilon\mathrm{(3S)}}}
\newcommand{\PgUd}{\ensuremath{\Upsilon\mathrm{(4S)}}}
\newcommand{\PgUe}{\ensuremath{\Upsilon\mathrm{(10860)}}}
\newcommand{\PgUf}{\ensuremath{\Upsilon\mathrm{(11020)}}}
\newcommand{\PgX}{\ensuremath{\Xi}}
\newcommand{\PgXa}{\ensuremath{\Xi\mathrm{(1530)P_{13}}}}
\newcommand{\PgXb}{\ensuremath{\Xi\mathrm{(1690)}}}
\newcommand{\PgXc}{\ensuremath{\Xi\mathrm{(1820)D_{13}}}}
\newcommand{\PgXd}{\ensuremath{\Xi\mathrm{(1950)}}}
\newcommand{\PgXe}{\ensuremath{\Xi\mathrm{(2030)}}}
\newcommand{\PgXm}{\ensuremath{\Xi^\mathrm{-}}}
\newcommand{\PgXz}{\ensuremath{\overline{\Xi}^\mathrm{0}}}
\newcommand{\Pgfa}{\ensuremath{\phi\mathrm{(1680)}}}
\newcommand{\Pgfiii}{\ensuremath{\phi_\mathrm{3}\mathrm{(1850)}}}
\newcommand{\Pgf}{\ensuremath{\phi\mathrm{(1020)}}}
\newcommand{\Pgg}{\ensuremath{\gamma}}
\newcommand{\Pgha}{\ensuremath{\eta\mathrm{(1295)}}}
\newcommand{\Pghb}{\ensuremath{\eta\mathrm{(1440)}}}
\newcommand{\Pghpr}{\ensuremath{\eta^{'}\mathrm{(958)}}}%\prime
\newcommand{\Pgh}{\ensuremath{\eta}}
\newcommand{\Pgmm}{\ensuremath{{\mu^-}}}
\newcommand{\Pgmp}{\ensuremath{{\mu^+}}}
\newcommand{\Pgm}{\ensuremath{{\mu}}}
\newcommand{\Pgne}{\ensuremath{\nu_\mathrm{{e}}}}
\newcommand{\Pgngm}{\ensuremath{\nu_{\mu}}}
\newcommand{\Pgngt}{\ensuremath{\nu_{\tau}}}
\newcommand{\Pgoa}{\ensuremath{\omega\mathrm{(1390)}}}
\newcommand{\Pgob}{\ensuremath{\omega\mathrm{(1600)}}}
\newcommand{\Pgoiii}{\ensuremath{\omega_\mathrm{3}\mathrm{(1670)}}}
\newcommand{\Pgo}{\ensuremath{\omega\mathrm{(783)}}}
\newcommand{\Pgpa}{\ensuremath{\pi\mathrm{(1300)}}}
\newcommand{\Pgpii}{\ensuremath{\pi_\mathrm{2}\mathrm{(1670)}}}
\newcommand{\Pgpm}{\ensuremath{\pi^-}}
\newcommand{\Pgppm}{\ensuremath{\pi^\mathrm{{\pm }}}}
\newcommand{\Pgpp}{\ensuremath{\pi^+}}
\newcommand{\Pgpz}{\ensuremath{\pi^0}}
\newcommand{\Pgp}{\ensuremath{\pi}}
\newcommand{\Pgra}{\ensuremath{\rho\mathrm{(1450)}}}
\newcommand{\Pgrb}{\ensuremath{\rho\mathrm{(1700)}}}
\newcommand{\Pgriii}{\ensuremath{\rho_\mathrm{3}\mathrm{(1690)}}}
\newcommand{\Pgr}{\ensuremath{\rho\mathrm{(770)}}}
\newcommand{\Pgt}{\ensuremath{\tau}}
\newcommand{\Pgya}{\ensuremath{\psi\mathrm{(3770)}}}
\newcommand{\Pgyb}{\ensuremath{\psi\mathrm{(4040)}}}
\newcommand{\Pgyc}{\ensuremath{\psi\mathrm{(4160)}}}
\newcommand{\Pgyd}{\ensuremath{\psi\mathrm{(4415)}}}
\newcommand{\Pgy}{\ensuremath{\psi\mathrm{(2S)}}}
\newcommand{\Phia}{\ensuremath{\mathrm{h_1(1170)}}}
\newcommand{\Pn}{\ensuremath{\mathrm{n}}}
\newcommand{\Pp}{\ensuremath{\mathrm{p}}}
\newcommand{\Pqb}{\ensuremath{\mathrm{q_b}}}
\newcommand{\Pqc}{\ensuremath{\mathrm{q_c}}}
\newcommand{\Pqd}{\ensuremath{\mathrm{q_d}}}
\newcommand{\Pqs}{\ensuremath{\mathrm{q_s}}}
\newcommand{\Pqt}{\ensuremath{\mathrm{q_t}}}
\newcommand{\Pqu}{\ensuremath{\mathrm{q_u}}}
\newcommand{\Pq}{\ensuremath{\mathrm{q}}}
\newcommand{\PsDipm}{\ensuremath{\mathrm{D_{s1}(2536)^{\pm}}}}
\newcommand{\PsDm}{\ensuremath{\mathrm{D_{s}^-}}}
\newcommand{\PsDp}{\ensuremath{\mathrm{D_{s}^+}}}
\newcommand{\PsDst}{\ensuremath{\mathrm{D_{s}^{\ast}}}}
\endinput


% Table of contents for each chapter

\usepackage[nottoc, notlof, notlot]{tocbibind}
\usepackage[english]{minitoc}
\setcounter{minitocdepth}{2}
\mtcindent=15pt
% Use \minitoc where to put a table of contents

\usepackage{aecompl}
\usepackage{lipsum}
% Glossary / list of abbreviations



\usepackage[intoc]{nomencl}
\renewcommand{\nomname}{Liste des Abréviations}

\makenomenclature

% My pdf code
% commenter pour usage avec xelatex

\usepackage{ifpdf}

\ifpdf
  \usepackage[pdftex]{graphicx}
  \DeclareGraphicsExtensions{.jpg,.png,.pdf}
  \usepackage[a4paper,hyperindex=true]{hyperref}
\else
  \usepackage{graphicx}
  \DeclareGraphicsExtensions{.ps,.eps}
  \usepackage[a4paper,dvipdfm,hyperindex=true]{hyperref}
\fi

\graphicspath{{.}{images/}}

\usepackage{subcaption}


% %nicer backref links
% \renewcommand*{\backref}[1]{}
% \renewcommand*{\backrefalt}[4]{%
% \ifcase #1 %
% (Non cité.)%
% \or
% (Cité en page~#2.)%
% \else
% (Cité en pages~#2.)%
% \fi}
% \renewcommand*{\backrefsep}{, }
% \renewcommand*{\backreftwosep}{ et~}
% \renewcommand*{\backreflastsep}{ et~}

% Links in pdf
\usepackage{color}
\definecolor{linkcol}{rgb}{0,0,0.4} 
\definecolor{citecol}{rgb}{0.5,0,0} 

% Change this to change the informations included in the pdf file

\hypersetup
{
bookmarksopen=true,
pdftitle={Thèse Nicolas Filipovic},
pdfauthor={Nicolas FILIPOVIC}, %auteur du document
pdfsubject={Measurements of $\Upsilon$ meson suppression in heavy ion collisions with the CMS experiment at the LHC}, %sujet du document
%pdftoolbar=false, %barre d'outils non visible
pdfmenubar=true, %barre de menu visible
pdfhighlight=/O, %effet d'un clic sur un lien hypertexte
colorlinks=false, %couleurs sur les liens hypertextes
pdfpagemode=None, %aucun mode de page
pdfpagelayout=SinglePage, %ouverture en simple page
pdffitwindow=true, %pages ouvertes entierement dans toute la fenetre
linkcolor=black, %couleur des liens hypertextes internes
citecolor=black, %couleur des liens pour les citations
urlcolor=black %couleur des liens pour les url
}

% definitions.
% -------------------

\setcounter{secnumdepth}{3}
\setcounter{tocdepth}{2}

% Some useful commands and shortcut for maths:  partial derivative and stuff

\newcommand{\pd}[2]{\frac{\partial #1}{\partial #2}}
\def\abs{\operatorname{abs}}
\def\argmax{\operatornamewithlimits{arg\,max}}
\def\argmin{\operatornamewithlimits{arg\,min}}
\def\diag{\operatorname{Diag}}
\newcommand{\eqRef}[1]{(\ref{#1})}

\usepackage{rotating}                    % Sideways of figures & tables
%\usepackage{bibunits}
%\usepackage[sectionbib]{chapterbib}          % Cross-reference package (Natural BiB)
%\usepackage{natbib}                  % Put References at the end of each chapter
                                         % Do not put 'sectionbib' option here.
                                         % Sectionbib option in 'natbib' will do.
\usepackage{fancyhdr}                    % Fancy Header and Footer

% \usepackage{txfonts}                     % Public Times New Roman text & math font
  
%%% Fancy Header %%%%%%%%%%%%%%%%%%%%%%%%%%%%%%%%%%%%%%%%%%%%%%%%%%%%%%%%%%%%%%%%%%
% Fancy Header Style Options

\pagestyle{fancy}                       % Sets fancy header and footer
\let\chaptermarkOld\chaptermark
\renewcommand{\chaptermark}[1]{\chaptermarkOld{#1}\sectionmark{}}
% \renewcommand{\chaptermark}[1]{\markright{#1}{}}
\fancyfoot{}                            % Delete current footer settings

%\renewcommand{\chaptermark}[1]{         % Lower Case Chapter marker style
%  \markboth{\chaptername\ \thechapter.\ #1}}{}} %

%\renewcommand{\sectionmark}[1]{         % Lower case Section marker style
%  \markright{\thesection.\ #1}}         %

\fancyhead[LE,RO]{\bfseries\thepage}    % Page number (boldface) in left on even
% pages and right on odd pages
\fancyhead[RE]{\bfseries\nouppercase{\leftmark}}      % Chapter in the right on even pages
\fancyhead[LO]{\bfseries\nouppercase{\rightmark}}     % Section in the left on odd pages

\let\headruleORIG\headrule
\renewcommand{\headrule}{\color{black} \headruleORIG}
\renewcommand{\headrulewidth}{1.0pt}
\usepackage{colortbl}
\arrayrulecolor{black}

\fancypagestyle{plain}{
  \fancyhead{}
  \fancyfoot{}
  \renewcommand{\headrulewidth}{0pt}
}

\usepackage{tikz}
\usetikzlibrary{calc}


\usepackage{MyAlgorithm}
\usepackage[noend]{MyAlgorithmic}

%%% Clear Header %%%%%%%%%%%%%%%%%%%%%%%%%%%%%%%%%%%%%%%%%%%%%%%%%%%%%%%%%%%%%%%%%%
% Clear Header Style on the Last Empty Odd pages
\makeatletter

\def\cleardoublepage{\clearpage\if@twoside \ifodd\c@page\else%
  \hbox{}%
  \thispagestyle{empty}%              % Empty header styles
  \newpage%
  \if@twocolumn\hbox{}\newpage\fi\fi\fi}

\makeatother
 
%%%%%%%%%%%%%%%%%%%%%%%%%%%%%%%%%%%%%%%%%%%%%%%%%%%%%%%%%%%%%%%%%%%%%%%%%%%%%%% 
% Prints your review date and 'Draft Version' (From Josullvn, CS, CMU)
\newcommand{\reviewtimetoday}[2]{\special{!userdict begin
    /bop-hook{gsave 20 710 translate 45 rotate 0.8 setgray
      /Times-Roman findfont 12 scalefont setfont 0 0   moveto (#1) show
      0 -12 moveto (#2) show grestore}def end}}
% You can turn on or off this option.
% \reviewtimetoday{\today}{Draft Version}
%%%%%%%%%%%%%%%%%%%%%%%%%%%%%%%%%%%%%%%%%%%%%%%%%%%%%%%%%%%%%%%%%%%%%%%%%%%%%%% 

\newenvironment{maxime}[1]
{
\vspace*{0cm}
\hfill
\begin{minipage}{0.5\textwidth}%
%\rule[0.5ex]{\textwidth}{0.1mm}\\%
\hrulefill $\:$ {\bf #1}\\
%\vspace*{-0.25cm}
\it 
}%
{%

\hrulefill
\vspace*{0.5cm}%
\end{minipage}
}

\let\minitocORIG\minitoc
\renewcommand{\minitoc}{\minitocORIG \vspace{1.5em}}

\usepackage{multirow}
% \usepackage{slashbox}
\usepackage{diagbox}

\newenvironment{bulletList}%
{ \begin{list}%
	{$\bullet$}%
	{\setlength{\labelwidth}{25pt}%
	 \setlength{\leftmargin}{30pt}%
	 \setlength{\itemsep}{\parsep}}}%
{ \end{list} }

\newtheorem{definition}{Définition}
\renewcommand{\epsilon}{\varepsilon}

% centered page environment

\newenvironment{vcenterpage}
{\newpage\vspace*{\fill}\thispagestyle{empty}\renewcommand{\headrulewidth}{0pt}}
{\vspace*{\fill}}

% feynmp
\usepackage{feynmp}
\DeclareGraphicsRule{*}{mps}{*}{}

% boldmath
\usepackage{bm}

% mpic
\usepackage[metapost]{mfpic}

% pstricks
% \usepackage{auto-pst-pdf}
% % \usepackage[pdf]{pstricks}
% \usepackage{pst-all}
% \usepackage{pstricks-add}

