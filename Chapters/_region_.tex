\message{ !name(Resume.tex)}
\message{ !name(Resume.tex) !offset(-2) }
\chapter*{Résumé en français}
\fancyhf{}
\selectlanguage{french}

Ce document, intitulé \textit{Mesures de la suppression des mésons
  $\Upsilon$~en collisions d'ions lourds dans l'expérience CMS au LHC},
est une thèse rédigée en vue d'obtenir le grade de Docteur de
l'Université Paris-Saclay, spécialité Physique Hadronique. Cette thèse
est le résumé de trois années de recherche passées au sein du
Laboratoire Leprince-Ringuet à l'École polytechnique à Palaiseau, sous
la direction du directeur de recherche Raphaël Granier de
Cassagnac. La présente thèse étant rédigée en anglais, elle nécessite
l'addition d'un \og résumé substantiel en français \fg{}, selon la procédure de
soutenance de doctorat rédigée par
l'Université\cite{saclaycons}. S'ensuit donc ce résumé dans le résumé.


\vspace{2.5em}
La présente thèse relate de mesures de la suppression
des mésons~\PgU\ en collisions d'ions plomb-plomb, relativement
à leur production en collisions proton-proton, à l'énergie dans le
centre de masse de \s\ = 2.76 TeV par paires de nucléons. Ces
collisions furent délivrées par le Grand collisionneur de hadrons
(LHC) au détecteur CMS en 2011 et 2013. Les mesures
effectuées précédemment sur le même
sujet~\cite{torsten,HIN-11-007,11-011} ont attesté de la quasi-complète
suppression des états excités \PgUb\ et \PgUc, ainsi que de la
nette modification du taux de production de l'état fondamental \PgUa,
en fonction de la centralité
de la collision plomb-plomb. Ces observations sont compatibles avec
l'hypothèse de fonte séquentielle des quarkonia lors de la formation
d'un plasma de quarks et de gluons (PQG).

Cette nouvelle analyse dispose d'un plus grand échantillon de
données proton-proton, ainsi que d'une meilleure reconstruction des
particules produites dans les collisions plomb-plomb. Avec ces deux
améliorations, il est possible d'étudier plus en détail le mécanisme
responsable de la suppression des quarkonia en présence du PQG. En
particulier, l'étude présentée relate des taux de production en fonction de variables
cinématiques telles que l'impulsion transverse et la rapidité, en plus d'une réitération de l'analyse en
fonction de la centralité.

Dans le premier chapitre, certaines notions phénoménologiques nécessaires à la
compréhension de la formation du PQG sont présentées. On fait d'abord
un point historique sur l'avènement de la chromodynamique quantique,
théorie quantique formalisant l'interaction forte entre quarks et
gluons, dont le confinement et la liberté asymptotique sont deux
propriétés bien connues. On se propose ensuite de monter en
température, pour laisser apparaître le caractère déconfiné de la
matière nucléaire portée à haute densité d'énergie. On énonce quelques
conséquences de cette transition de phase qu'est le PQG sur les particules produites
dans les collisions nucléaires ultrarelativistes. En particulier, la
fonte séquentielle des quarkonia est, historiquement, une des preuves
irréfutables de la formation du PQG. Le second chapitre se propose de
décrire les quarkonia, autant du point de vue de la
production en collisions proton-proton que dans le contexte de
collisions nucléaires. Leurs mécanismes de production étant encore
aujourd'hui disputés, il n'est pas possible de conclure sur ce
sujet dans le contexte de cette thèse. Cependant, une mesure de leur
modification dans la matière nucléaire est possible dès lors qu'on
dispose d'une référence, qui est mesuré expérimentalement en
collisions proton-proton. La partie théorique de cette thèse se
termine par un résumé de l'état de l'art de la recherche sur la
suppression des quarkonia avant le début de cette thèse.

La deuxième partie du présent rapport est dédiée au contexte
expérimental. Elle résume en deux chapitres les dispositifs
expérimentaux nécessaires à la production des données ci-analysées,
ainsi que les outils matériels et logiciels employés dans la présente
analyse. Le troisième chapitre contient d'abord une description succinte du LHC et
de sa capacité à produire des collisions ultrarelativistes de
différent noyaux atomiques. Les détecteurs collectant des données auprès
du LHC sont mentionnés, avant de présenter en détail le détecteur
utilisé dans l'analyse. Le Solénoïde compact à muons (CMS) est une
expérience polyvalente, capable de mesurer en détail des processus
rares tel que le mécanisme de Higgs, ainsi que d'enregistrer
l'ensemble des objets produits dans une collision d'ions lourds. Les
moyens mis en place pour déclencher la prise de données doivent donc
être flexibles et précis. 
Le quatrième chapitre présente les spécificités et les performances
nécessaires au bon déroulement de l'analyse. Le système de
déclenchement du détecteur est présenté, ainsi que ses performances
lors des prises de données auxquelles j'ai assisté en
2013. Les différentes méthodes de reconstruction et de sélections des muons, ces leptons fugaces que le détecteur reconnaît si
bien, sont également présentées dans le chapitre 4. Enfin, la
catégorisation des collisions d'ions lourds en plusieurs classes de
centralités est présentée. Cet aspect important des collisions d'ions
lourds est indispensable pour bien identifier les collisions dans
lesquelles le PQG est susceptible d'être créé.

La troisième partie, dédiée à l'analyse proprement dite, est séparée
en trois chapitres. Le chapitre cinq représente probablement le c\oe{}ur
du travail effectué ces trois dernières années. Dans ce chapitre, on
se propose de revisiter la sélection cinématique des muons découlant
de la désintégration des \PgU, afin d'améliorer le rapport signal sur
bruit dans l'analyse. On découvre que les résonances étant produites
dans un régime non relativiste, la désintégration en deux muons est
nettement asymmétrique. Cette particularité permet de récupérer près
de trente-cinq pourcent de données de bonne qualité, autrement
perdues si une sélection plus standard eût été appliquée. Enfin,
l'extraction du signal est détaillée, ainsi que les taux de signal
résultant en fonction des variables cinématiques \pt, \y, et de la
centralité de la collisions plomb-plomb. Les variations nécessaires à
l'établissement d'une erreur systématique sur l'extraction du signal
sont présentées. Ces mesures permettront l'expression des
résultats au chapitre sept, sous la forme de sections efficaces de production en collisions
proton-proton, des taux corrigés en
collisions plomb-plomb, ainsi que du facteur de modification nucléaire
(\RAA) pour chaque méson, en fonction de l'observable considérée.
Le sixième chapitre établit comment les taux bruts enregistrés au
chapitre cinq sont corrigés de certains effets naturels de perte du
signal. L'acceptance et l'efficacité de détection sont définies et
calculées pour les trois états \PgU(1S,2S,3S). Différentes sources d'erreur
connues dans l'estimation de l'efficacité doivent être évaluées
précisément, au regard de la méthode de déclenchement et de la méthode
de sélection utilisées, afin de rendre à l'efficacité de détection une
valeur corrigée. Une méthode célèbre dans ce type d'analyse est
présentée. Enfin, les erreurs systématiques relatives à chaque
correction (acceptance, efficacité, correction de second ordre) sont
estimées.
Le chapitre sept présente les résultats obtenus grâce aux mesures et à
leur corrections successives. Les sections efficaces de production en
collisions proton-proton sont tout d'abord présentées pour les mésons
\PgUa, \PgUb\ et \PgUc, en fonction de leur impulsion transverse et de
leur rapidité. Des mesures équivalentes sont présentées ensuite dans
le contexte des collisions plomb-plomb, afin d'extraire le facteur de
modification nucléaire. On observe que les \RAA\ des \PgU(1S,2S,3S)
analysés ici sont compatibles et plus précis que les précédents
résultats exprimés par CMS en 2012 à la même énergie dans le centre de
masse~\cite{torsten,HIN-11-007,11-011}. Ces résultats montrent
également que la suppression des \PgUa\ et \PgUb\ est indépendante de
l'impulsion transverse à laquelle ces mésons sont produits. L'état
excité \PgUc\ est encore insaisissable en collisions plomb-plomb, car
fortement supprimé : on établit alors une limite supérieure à sa
suppression, dans l'attente d'une mesure ultérieure, plus
significative. Le \RAA\ mesuré pour le \PgUa\ en fonction de la
rapidité est aussi compatible avec une mesure de l'expérience ALICE~\cite{ALICEUpsilonHI},
et laisse à croire que la suppression de ces derniers est
essentiellement indépendante de la rapidité. Pour finir, les mesures
de la présente analyse sont comparées à des modèles phénoménologiques
de suppression des quarkonia dans le PQG. Dans la majeure partie de
l'espace des paramètres considéré, les modèles reproduisent
correctement les données.

\message{ !name(Resume.tex) !offset(-147) }
